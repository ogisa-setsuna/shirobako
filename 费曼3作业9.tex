\documentclass[utf8]{ctexart}
\usepackage{graphicx}
\usepackage{amsmath}
\usepackage{braket}
\usepackage{amssymb}
\usepackage{bm}
\usepackage{siunitx}
\usepackage{float}
\makeatletter
\newcommand{\rmnum}[1]{\romannumeral #1}
\newcommand{\Rmnum}[1]{\expandafter\@slowromancap\romannumeral #1@}
\makeatother
\title{费曼3作业9}
\author{郑子诺,物理41}
\date{\today}
\begin{document}
\maketitle
\noindent82.1\\
(a)Neglect the interaction between the electrons, then we just need solve the one-electron Schrodinger equations, where the matrix of Hamiltonian is
\[\begin{bmatrix}
	E_0&-A&0&0\\
	-A&E_0&-A&0\\
	0&-A&E_0&-A\\
	0&0&-A&E_0
\end{bmatrix}\]
Solve the characteristic equation we have
\[E=E_0\pm\sqrt{\frac{3+\sqrt{5}}{2}}A,E_0\pm\sqrt{\frac{3-\sqrt{5}}{2}}A\]
The energy gap between the first excited state and the ground state is just $2\sqrt{\dfrac{3-\sqrt{5}}{2}}A\approx1.236\unit{eV}$, and therefore the wavelength $\lambda$ is
\[\lambda\approx1003\unit{nm}\]
(b)Solve the equations, we know that the distribution is just $C\sin kx$. The two low energies correspond to $k=\dfrac{\pi}{5},\dfrac{2\pi}{5}$. Thus Two electrons will have the same distribution which spreads like a wave packet, and the rest electron will spread symmetrically about the center of the molecule.\\
83.1\\
(a)If not, the left side of the Schrodinger equation
\[-\frac{\hbar^2}{2m}\frac{\mathrm{d^2}u_0(x)}{\mathrm{d}x^2}+V_0u_0(x)=Eu_0(x)\]
will diverge. Hence we need $u_0(x)=0$ for $x<0,x>a$.\\
(b)We have
\[u(x)=C_1\cos kx+C_2\sin kx,k=\sqrt{\frac{2mE}{\hbar^2}}\]
according to the condition above and the normalization condition, we have
\[u(x)=\sqrt{\frac{2}{a}}\sin kx,k=\frac{n\pi}{a},n=1,2,\dots\]
(c)
\[E_0=\frac{\pi^2\hbar^2}{2ma^2},u_0=\sqrt{\frac{2}{a}}\sin\frac{\pi}{a}x\]
$u_0(x)$ is just like a wave packet.\\
(d)
\[\Delta E=E_1-E_0=\frac{3\pi^2\hbar^2}{2ma^2}\]
(e)
\begin{figure}[H]
	\centering
	\includegraphics[width=0.5\textwidth]{9.jpg}
\end{figure}
83.2\\
(a)Solve the Schrodinger equation, we have
\begin{align*}
	\psi(x)&=Ae^{\beta x}+B^{-\beta x},x>a\\
	\psi(x)&=C_1\cos\alpha x+C_2\sin\alpha x,-a<x<a\\
	\psi(x)&=Ce^{\beta x}+De^{-\beta x}
\end{align*}
According to the boundary conditions, we have
\[A=D=0\]
\[C_1\cos\alpha a+C_2\sin\alpha a=Be^{-\beta a},-\alpha C_1\sin\alpha a+\alpha C_2\cos\alpha a=-\beta Be^{-\beta a}\]
\[C_1\cos\alpha a-C_2\sin\alpha a=Ce^{-\beta a},\alpha C_1\sin\alpha a+\alpha C_2\cos\alpha a=\beta Ce^{-\beta a}\]
Then we have
\[\alpha\tan\alpha a=\beta\]
or
\[\alpha\cot\alpha a=-\beta\]
(b)
The equation now is
\[y\tan y=\sqrt{4-y^2},y=\alpha a\]
or
\[y\cot y=-\sqrt{4-y^2}\]
Therefore we have
\[E_0\approx1.03\frac{\hbar^2}{2ma},E_1\approx1.90\frac{\hbar^2}{2ma}\]
\begin{figure}[H]
	\centering
	\includegraphics[width=0.7\textwidth]{99.jpg}
\end{figure}
(c)Only one. Because $\sqrt{\dfrac{1}{4}-y^2}$ will be zero before $\dfrac{\pi}{2}$.
\end{document}














