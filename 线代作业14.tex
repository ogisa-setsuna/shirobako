\documentclass[utf8]{ctexart}
\usepackage{graphicx}
\usepackage{amsmath}
\usepackage{amssymb}
\usepackage{braket}
\title{线代作业14}
\author{郑子诺,物理41}
\date{\today}
\begin{document}
\maketitle
\noindent1.\\
\[\braket{a\alpha+b\beta|a\alpha+b\beta}=\|a\alpha\|^2+\|b\beta\|^2+2\mathbb{Re}\{\overline{a}b\braket{\alpha|\beta}\}\]
若$\alpha,\beta$正交,显然有
\[\braket{a\alpha+b\beta|a\alpha+b\beta}=\|a\alpha\|^2+\|b\beta\|^2\]
反之,上式成立,我们有
\[\mathbb{Re}\{\overline{a}b\braket{\alpha|\beta}\}=0\]
令$\braket{\alpha|\beta}=u+iv$。令$a=b=1$,我们有$u=0$,令$a=1,b=i$,我们有$v=0$,因此$\alpha,\beta$正交。\\
2.\\
利用爱因斯坦求和约定,我们有
\[\braket{A|B}=Tr(\overline{A}^TB)=\overline{a}_{ki}b_{ki}\]
显然满足准线性条件和共轭对称性,$A=B$时,恰是每个矩阵元的模方,因而满足正定性。\\
3.\\
我们有
\[Tr(AB)^2=a_{ik}b_{kl}a_{lk'}b_{k'i},Tr(A^2B^2)=a_{ik}a_{kl}b_{lk'}b_{k'i}\]
取共轭并利用厄米性得到
\[\overline{Tr(AB)^2}=a_{ki}b_{ik'}a_{k'l}b_{lk},\overline{Tr(A^2B^2)}=a_{lk}a_{ki}b_{ik'}b_{k'l}\]
更换哑指标名称可见其相同,因而都是实数。鉴于求迹的可交换性以及转置不变性加之迹为实数,我们有
\begin{align*}
	Tr(A^2B^2)-Tr(AB)^2&=Tr(BA(AB-BA))\\
	&=-Tr(AB(AB-BA))\\
	&=\frac{1}{2}Tr[(BA-AB)(AB-BA)]\\
	&=\frac{1}{2}Tr[(AB-BA)^\dagger(AB-BA)]\ge0
\end{align*}
等号成立当且仅当$AB=BA$。\\
4.\\
我们知道$\det U=e^{i\theta}$,因此总有
\[U=\begin{bmatrix}
	a&b\\
	c&d
\end{bmatrix}\begin{bmatrix}
e^{i\frac{\theta}{2}}&\\
&e^{i\frac{\theta}{2}}
\end{bmatrix}\]
其中$\begin{vmatrix}
	a&b\\
	c&d
\end{vmatrix}=1$且是酉矩阵。因此我们有
\[ad-bc=1,a\overline{a}+c\overline{c}=1,a\overline{b}+c\overline{d}=0\]
可以解得
\[c=-\overline{b},d=\overline{a}\]
于是我们有
\[U=\begin{bmatrix}
	a&b\\
	-\overline{b}&\overline{a}
\end{bmatrix}\begin{bmatrix}
e^{i\frac{\theta}{2}}&\\
&e^{i\frac{\theta}{2}}
\end{bmatrix}\]
令$a=a'e^{i\alpha_1},b=b'e^{i\alpha_2}$。
令
\[\theta_1=\alpha_1+\frac{\theta}{2},\theta_2=-\alpha_2+\frac{\theta}{2},\theta_3=0,\theta_4=\alpha_2-\alpha_1\]
我们总可以将一个酉矩阵写成
\[U=\begin{bmatrix}
e^{i\theta_1}&\\
&e^{i\theta_2}
\end{bmatrix}\begin{bmatrix}
a'&b'\\
-b'&a'
\end{bmatrix}\begin{bmatrix}
e^{i\theta_3}&\\
&e^{i\theta_4}
\end{bmatrix}\]
显然中间矩阵为实正交矩阵,因而可以写作
\[U=\begin{bmatrix}
	e^{i\theta_1}&\\
	&e^{i\theta_2}
\end{bmatrix}\begin{bmatrix}
	\cos\theta&-\sin\theta\\
	\sin\theta&\cos\theta
\end{bmatrix}\begin{bmatrix}
	e^{i\theta_3}&\\
	&e^{i\theta_4}
\end{bmatrix}\]
5.\\
(a)根据分块矩阵的性质,我们显然有
\[\varphi(A_1)\varphi(A_2)=\begin{bmatrix}
	B_1B_2-C_1C_2&B_1C_2+C_1B_2\\
	-B_1C_2-C_1B_2&B_1B_2-C_1C_2
\end{bmatrix}\]
同时我们有
\[A_1A_2=B_1B_2-C_1C_2+i(B_1C_2+C_1B_2)\]
因此$\varphi(A_1A_2)=\varphi(A_1)\varphi(A_2)$。显然我们有
\[\overline{A}^T=B^T-iC^T\]
因此
\[\varphi(\overline{A}^T)=\begin{bmatrix}
	B&-C\\
	C&B
\end{bmatrix}=\varphi(A)^T\]
(b)若$A$是厄米矩阵,则有$\varphi(A^\dagger)=\varphi(A)=\varphi(A)^T$,因而$\varphi(A)$对称。反之我们有$\varphi(A^\dagger)=\varphi(A)$,显然$\varphi(A)=\varphi(A')$当且仅当$A=A'$,因此$A=A^\dagger$,$A$为厄米矩阵。\\
(c)若$A$是酉矩阵,则$\varphi(I)=\varphi(AA^\dagger)=\varphi(A)\varphi(A^\dagger)=\varphi(A)\varphi(A)^T=I$,于是$\varphi(A)$为正交矩阵。反之我们有$\varphi(AA^\dagger)=I$,显然有$AA^\dagger=I$,因此$A$为酉矩阵。\\
6.\\
若$B,D$是酉矩阵,$C=0$,那么$A$显然是酉矩阵,因为我们有
\[AA^\dagger=\begin{bmatrix}
	BB^\dagger+CC^\dagger&CD^\dagger\\
	DC^\dagger&DD^\dagger
\end{bmatrix}=I\]
反之,根据行列式不为$0$我们知道$B,D$可逆,因此一定有$C=0$,紧接着$BB^\dagger=I,DD^\dagger=I$,因而都是酉矩阵。\\
7.\\
应用我们曾经证明过的结论,任何一个复方阵酉相似于一个上三角矩阵,显然其对角线正是其特征值。显然我们有
\[Tr(A^TA)=|\lambda_1|^2+|a_{12}|^2+|\lambda_2|^2+\cdots+|a_{1n}|^2+\cdots+|\lambda_n|^2\ge\sum_{i=1}^n|\lambda_i|^2\]
等式成立当且仅当上三角矩阵是对角矩阵,因而$A$可对角化,于是$A$是正规矩阵。\\
8.\\
若$A$正定,定义$P=\sqrt{A}$,显然有$A=P^\dagger P$,且由于特征值非零$P$一定可逆。反之显然有
\[\braket{x|Ax}=\braket{Px|Px}\ge0\]
且等于零当且仅当$Px=0$,由于$P$可逆,一定有$x=0$,因而正定。\\
9.\\
\[\begin{bmatrix}
	1&1&1\\
	1&1&-1\\
	-2&1&0
\end{bmatrix}=\begin{bmatrix}
\dfrac{1}{\sqrt{6}}&1&\dfrac{1}{\sqrt{2}}\\[8pt]
\dfrac{1}{\sqrt{6}}&1&-\dfrac{1}{\sqrt{2}}\\[8pt]
-\dfrac{2}{\sqrt{6}}&1&0\\
\end{bmatrix}\begin{bmatrix}
\sqrt{6}&&\\
&1&\\
&&\sqrt{2}
\end{bmatrix}\]
\end{document}

















