\documentclass[utf8]{ctexart}
\usepackage{graphicx}
\usepackage{amsmath}
\usepackage{amssymb}
\title{线代作业8}
\author{郑子诺,物理41}
\date{\today}
\begin{document}
\maketitle
\noindent1.\\
(a)\[\lambda^3+3\lambda^2-\lambda-5=0\]
特征值为
\[\lambda=r\cos\theta-1,r\cos(\theta+\frac{2\pi}{3})-1,r\cos(\theta+\frac{4\pi}{3})-1\]
\[r=\frac{4}{\sqrt{3}},\theta=\frac{1}{3}\arccos\frac{3\sqrt{3}}{8}\]
对应的特征向量为
\[\alpha=c\begin{bmatrix}
	\dfrac{2\lambda+3}{\lambda^2+\lambda-1}\\[8pt]
	\dfrac{3\lambda+4}{\lambda^2+\lambda-1}\\[8pt]
	1
\end{bmatrix}\]
(b)
\[\lambda^3-2\lambda^2+1=0\]
特征值为
\[\lambda=1,\frac{\sqrt{5}+1}{2},\frac{-\sqrt{5}+1}{2}\]
对应的特征向量分别为
\[\alpha_1=c\begin{bmatrix}
	0\\[8pt]
	1\\[8pt]
	0
\end{bmatrix},
\alpha_2=c\begin{bmatrix}
	\dfrac{\sqrt{5}-1}{2}\\[8pt]
	0\\[8pt]
	1
\end{bmatrix},
\alpha_3=c\begin{bmatrix}
	-\dfrac{\sqrt{5}+1}{2}\\[8pt]
	0\\[8pt]
	1
\end{bmatrix}\]
(c)
令$f(x)=a_0+a_1x+\cdots+a_{n-1}x^{n-1}$
\[(\lambda-f(1))(\lambda-f(\omega))\cdots(\lambda-f(\omega^{n-1}))=0\]
\[\lambda=f(1),f(\omega),\dots,f(\omega^{n-1})\]
\[\omega=\cos\frac{2\pi}{n}+i\sin\frac{2\pi}{n}\]
\[\alpha_k=\begin{bmatrix}
	1\\
	\omega^{k}\\
	\vdots\\
	\omega^{k(n-1)}
\end{bmatrix},0\le k\le n-1\]
2.\\
\[\lambda=2,4\]
\[\alpha_1=\frac{1}{\sqrt{2}}\begin{bmatrix}
	1\\
	1\\
	0
\end{bmatrix},
\alpha_2=\frac{1}{\sqrt{2}}\begin{bmatrix}
	1\\
	0\\
	1
\end{bmatrix},
\alpha_3=\frac{1}{\sqrt{2}}\begin{bmatrix}
	0\\
	1\\
	1
\end{bmatrix}\]
\[P=\frac{1}{\sqrt{2}}\begin{bmatrix}
	1&1&0\\
	1&0&1\\
	0&1&1
\end{bmatrix}\]
\[P^{-1}=\frac{1}{\sqrt{2}}\begin{bmatrix}
	1&1&-1\\
	1&-1&1\\
	-1&1&1
\end{bmatrix}\]
\[A^{20}=P\begin{bmatrix}
	2^{20}&0&0\\
	0&2^{20}&0\\
	0&0&4^{20}
\end{bmatrix}P^{-1}=\begin{bmatrix}
2^{20}&0&0\\
2^{19}-2^{39}&2^{19}+2^{39}&-2^{19}+2^{39}\\
2^{19}-2^{39}&-2^{19}+2^{39}&2^{19}+2^{39}
\end{bmatrix}\]
3.\\
\[\det(\lambda I-A)=\det((\lambda A^{-1}-I)A)=(-1)^n\lambda^n\det A\det(\frac{1}{\lambda}I-A^{-1})\]
\[\therefore a'_k=\frac{a_{n-k}}{(-1)^n\det A}\]
对于每一个$A$的特征值有
\[\lambda'=\frac{1}{\lambda}\]
4.\\
\[Tr(AB)=\sum\limits_{l=1}^m\sum\limits_{k=1}^na_{lk}b_{kl}=\sum\limits_{k=1}^n\sum\limits_{l=1}^mb_{kl}a_{lk}=Tr(BA)\]
5.\\
\[Tr(AB-BA)=Tr(AB)-Tr(BA)=0\neq Tr(I_n)=n\]
6.\\
\[\begin{bmatrix}
	I&-A\\
	0&I
\end{bmatrix}\begin{bmatrix}
\lambda I&A\\
B&I
\end{bmatrix}=\begin{bmatrix}
\lambda I-AB&0\\
B&I
\end{bmatrix}\]
\[\begin{bmatrix}
	I&0\\
	-B&I
\end{bmatrix}\begin{bmatrix}
	\lambda I&A\\
	B&I
\end{bmatrix}=\begin{bmatrix}
	\lambda I&A\\
	0&I-BA
\end{bmatrix}\]
\[\therefore\det(\lambda I-AB)=\det(\lambda I-BA)\]
特征值相同。\\
7.\\
充分性是显然的。
\[\det(\lambda I-A)=\det(P^{-1}(\lambda I-B)P)=\det(\lambda I-B)\]
很容易发现,要使得一个对角矩阵的对角元素交换顺序,只需要在两边同时乘上一对相同的表示交换两行的初等行变换矩阵,显然两者互相为逆。那么我们有
\[B=E_1E_2\cdots E_nAE_n^{-1}\cdots E_2^{-1}E_1^{-1}=P^{-1}AP\]
8.\\
\[A=\begin{bmatrix}
	2&-2&0\\
	2&2&0\\
	0&0&0
\end{bmatrix},
B=\begin{bmatrix}
	4&0&0\\
	0&0&0\\
	0&0&0
\end{bmatrix}\]
两者特征方程显然相同,但是不相似。因为初等行列变换不改变矩阵的秩,每一个可逆矩阵又可以表示成初等行列变换的复合,这意味着相似的矩阵具有相同的秩,而$A,B$的秩显然不同,前者为$2$后者$1$。\\
9.\\
显然可以直接因式分解
\[\phi(B)=(B-\lambda_1I)(B-\lambda_2I)\cdots(B-\lambda_nI)\]
然后就再显然不过了,该是行列式不为$0$当且仅当$B$没有和$A$相同的特征值,这直接导出可逆性。\\
10.\\
显然有
\[A^nX=A^{n-1}XB=\cdots=XB^n\]
设$\phi(x)$为$A$的特征方程,那么有
\[\phi(A)X=X\phi(B)=0\]
又由于$\phi(B)$可逆,我们有
\[X=0\]
\end{document}



































