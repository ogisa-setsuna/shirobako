\documentclass[utf8]{ctexart}
\usepackage{graphicx}
\usepackage{amsmath}
\usepackage{amssymb}
\usepackage{siunitx}
\makeatletter
\newcommand{\rmnum}[1]{\romannumeral #1}
\newcommand{\Rmnum}[1]{\expandafter\@slowromancap\romannumeral #1@}
\makeatother
\title{统计力学作业3}
\author{郑子诺,物理41}
\date{\today}
\begin{document}
\maketitle
\noindent3.5\\
我们有
\[\left(\frac{\partial U}{\partial n}\right)_{T,V}=T\left(\frac{\partial S}{\partial n}\right)_{T,V}+\mu\]
根据$\mathrm{d}F=-S\mathrm{d}T-p\mathrm{d}V+\mu\mathrm{d}n$我们知道
\[\left(\frac{\partial S}{\partial n}\right)_{T,V}=-\left(\frac{\partial \mu}{\partial T}\right)_{V,n}\]
因此
\[\left(\frac{\partial U}{\partial n}\right)_{T,V}-\mu=-T\left(\frac{\partial \mu}{\partial T}\right)_{V,n}\]
4.2\\
根据
\[\mu_i=\left(\frac{\partial G}{\partial n_i}\right)_{n_{j\neq i}}\]
我们有
\begin{align*}
	&\sum_jn_j\frac{\partial \mu_i}{\partial n_j}\\[8pt]
	&=\frac{\partial}{\partial n_i}\left(\sum_jn_j\frac{\partial G}{\partial n_j}\right)-\frac{\partial G}{\partial n_i}\\[8pt]
	&=0
\end{align*}
4.8\\
(1)由于过程绝热,根据能量守恒温度不变,我们有
\[p\left(\frac{n_1RT}{p_1}+\frac{n_2RT}{p_2}\right)=(n_1+n_2)RT\]
因此
\[p=\frac{(n_1+n_2)p_1p_2}{n_1p_2+n_2p_1}\]
(2)由于是不同的气体,直接计算其最终态分压,得到熵。
\[p_1'=\frac{n_1p_1p_2}{n_1p_2+n_2p_1}\]
\[p_2'=\frac{n_2p_1p_2}{n_1p_2+n_2p_1}\]
\[\Delta S=n_1R\ln\frac{n_1p_2+n_2p_1}{n_1p_2}+n_2R\ln\frac{(n_1p_2+n_2p_1)}{n_2p_1}\]
(3)由于是同种气体,分别使其通过可逆过程达到最终态再叠加,得到
\[\Delta S=n_1R\ln\frac{n_1p_2+n_2p_1}{(n_1+n_2)p_2}+n_2R\ln\frac{(n_1p_2+n_2p_1)}{(n_1+n_2)p_1}\]
3.7\\
根据
\[L=T\Delta S_m=\Delta U_m+p\Delta V_m\]
以及克拉伯龙方程
\[\frac{\mathrm{d}p}{\mathrm{d}T}=\frac{L}{T\Delta V_m}\]
我们有
\[\Delta U_m=L\left(1-\frac{p}{T}\frac{\mathrm{d}T}{\mathrm{d}p}\right)\]
3.10\\
我们有
\[L=T\Delta S_m\]
对$T$求导,我们得到
\[\frac{\mathrm{d}L}{\mathrm{d}T}=T\left(\frac{\partial \Delta S_m}{\partial T}\right)_p+\frac{L}{T}+T\left(\frac{\partial \Delta S_m}{\partial p}\right)_T\frac{\mathrm{d}p}{\mathrm{d}T}\]
根据麦克斯韦关系和克拉伯龙方程
\[\left(\frac{\partial S}{\partial p}\right)_T=-\left(\frac{\partial V}{\partial T}\right)_p\]
\[\frac{\mathrm{d}p}{\mathrm{d}T}=\frac{L}{T\Delta V_m}\]
以及
\[c_p=T\left(\frac{\partial S_m}{\partial T}\right)_p\]
我们有
\[\frac{\mathrm{d}L}{\mathrm{d}T}=c_p^\beta-c_p^\alpha+\frac{L}{T}-T\left[\left(\frac{\partial V_m^\beta}{\partial T}\right)_p-\left(\frac{\partial V_m^\alpha}{\partial T}\right)_p\right]\frac{L}{V_m^\beta-V_m^\alpha}\]
若$\beta$为气态,$\alpha$为凝聚态,忽略凝聚态体积及其变化并带入理想气体方程得到
\[\frac{\mathrm{d}L}{\mathrm{d}T}=c_p^\beta-c_p^\alpha\]
3.13\\
我们有
\[\left(p+\frac{a}{V_m^2}\right)(V_m-b)=RT\]
求导并使其等于$0$得到
\[\frac{RT}{(V_m-b)^2}=\frac{2a}{V_m^3}\]
代入原方程消去$RT$,容易得到
\[pV_m^3=a(V_m-2b)\]
图中区域\Rmnum{1}和区域\Rmnum{3}为亚稳态区域,区域\Rmnum{2}为不稳定态区域。
\end{document}











