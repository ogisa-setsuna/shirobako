\documentclass[utf8]{ctexart}
\usepackage{graphicx}
\usepackage{amsmath}
\usepackage{amsthm}
\title{线代作业3}
\author{郑子诺,物理41}
\date{\today}
\begin{document}
\maketitle
\noindent1.\\
(1):\[\begin{bmatrix}
	11&7&5\\
	14&17&4\\
	11&8&12
\end{bmatrix}\]
(2):\[\begin{bmatrix}
	12&12\\
	3&8\\
	13&19
\end{bmatrix}\]
(3):\[\begin{bmatrix}
	5&6&7&8\\
	10&12&14&16\\
	15&18&21&24\\
	20&24&28&32
\end{bmatrix}\]
2.\\
显然,由于$AB=BA$,矩阵乘法运算将与实数无异,于是由二项式展开得:
\[(A+B)^k=\sum\limits_{i=0}^{k}\frac{k!}{i!(k-i)!}A^iB^{k-i}\]
若$AB\neq BA$,则显然该式不成立,例如:
\[(A+B)^2=A^2+AB+BA+B^2\neq A^2+2AB+B^2\]
3.\\
(1):\\
\[\begin{bmatrix}
	1&1\\
	0&1
\end{bmatrix}
\begin{bmatrix}
    1&1\\
    0&1
\end{bmatrix}
=
\begin{bmatrix}
	1&2\\
	0&1
\end{bmatrix}\]
\[\begin{bmatrix}
	1&k-1\\
	0&1
\end{bmatrix}
\begin{bmatrix}
	1&1\\
	0&1
\end{bmatrix}
=
\begin{bmatrix}
	1&k\\
	0&1
\end{bmatrix}\]
由归纳法可得:
\[A^n=
\begin{bmatrix}
	1&n\\
	0&1
\end{bmatrix}\]
(2):\\
\[\begin{bmatrix}
	a^{k-1}&(k-1)a^{k-2}b\\
	0&a^{k-1}
\end{bmatrix}
\begin{bmatrix}
	a&b\\
	0&a
\end{bmatrix}
=
\begin{bmatrix}
	a^k&ka^{k-1}b\\
	0&a^k
\end{bmatrix}\]
由归纳法得:
\[A^n=
\begin{bmatrix}
	a^n&na^{n-1}b\\
	0&a^n
\end{bmatrix}\]
(3):\\
\begin{align}
	\begin{bmatrix}
		\cos\theta&-\sin\theta\\
		\sin\theta&\cos\theta
	\end{bmatrix}
	\begin{bmatrix}
		\cos\alpha&-\sin\alpha\\
		\sin\alpha&\cos\alpha
	\end{bmatrix}
	&=
	\begin{bmatrix}
		\cos\theta\cos\alpha-\sin\theta\sin\alpha&-\sin\alpha\cos\theta-\sin\theta\cos\alpha\\
		\sin\alpha\cos\theta+\sin\theta\cos\alpha&\cos\theta\cos\alpha-\sin\theta\sin\alpha
	\end{bmatrix}\notag\\
	&=
	\begin{bmatrix}
		\cos(\alpha+\theta)&-\sin(\alpha+\theta)\\
		\sin(\alpha+\theta)&\cos(\alpha+\theta)
	\end{bmatrix}\notag
\end{align}
所以显然有:
\[A^n=
\begin{bmatrix}
	\cos n\theta&-\sin n\theta\\
	\sin n\theta&\cos n\theta
\end{bmatrix}\]
(4):\\
\[
\begin{bmatrix}
	a&b&c&d\\
	0&e&f&g\\
	0&0&h&i\\
	0&0&0&j
\end{bmatrix}
\begin{bmatrix}
	1&1&0&0\\
	0&1&1&0\\
	0&0&1&1\\
	0&0&0&1
\end{bmatrix}
=
\begin{bmatrix}
	a&a+b&b+c&c+d\\
	0&e&e+f&f+g\\
	0&0&h&h+i\\
	0&0&0&j
\end{bmatrix}\]
显然,矩阵元素构成杨辉三角,因此:
\[A^n=
\begin{bmatrix}
	1&n&\frac{n(n-1)}{2}&\frac{n(n-1)(n-2)}{6}\\
	0&1&n&\frac{n(n-1)}{2}\\
	0&0&1&n\\
	0&0&0&1
\end{bmatrix},n\ge2\]
(5):\\
显然我们有:
\[A=XY,
X=\\
\begin{bmatrix}
	a_1\\
	a_2\\
	a_3
\end{bmatrix},
Y=
\begin{bmatrix}
	b_1&b_2&b_3
\end{bmatrix}\]
\[YX=a_1b_1+a_2b_2+a_3b_3\]
因此:
\begin{align}
	A^n=(XY)^n&=X(YX)(YX)\cdots(YX)Y=(a_1b_1+a_2b_2+a_3b_3)^{n-1}XY\notag\\
	&=(a_1b_1+a_2b_2+a_3b_3)^{n-1}\begin{bmatrix}
		a_1b_1&a_1b_2&a_1b_3\\
		a_2b_1&a_2b_2&a_2b_3\\
		a_3b_!&a_3b_2&a_3b_3
	\end{bmatrix}\notag
\end{align}
4.\\
(1):\\
\[\begin{cases}
	x=5\\
	y=5\\
	z=2
\end{cases}\]
(2):\\
\[\begin{cases}
	x_1=1\\
	x_2=2\\
	x_3=3\\
	x_4=-1
\end{cases}\]
(3):\\
无解,因为原增广矩阵行等价于:
\[\begin{bmatrix}
	1&4&-2&3&5\\
	0&-6&1&-2&-9\\
	0&-13&8&-8&-6\\
	0&0&0&0&-2
\end{bmatrix}\]
5.\\
系数行列式为$\lambda^4-4\lambda^2$,所以:
当$\lambda=0$时,有无穷多个解:
\[\begin{cases}
	x_3=1-x_1\\
	x_4=1-x_2
\end{cases}\]
当$\lambda=2$时,有无穷多个解:
\[\begin{cases}
	x_1=x_3=\frac{1}{2}-x_4\\
	x_2=x_4
\end{cases}\]
当$\lambda=-2$时,无解,因为其增广矩阵行等价于:
\[\begin{bmatrix}
	-1&1&-1&1&1\\
	0&0&0&0&2\\
	0&1&-2&1&1\\
	1&0&1&-2&1
\end{bmatrix}\]
当$\lambda\neq0,2,-2$时,有唯一解为:
\[x_1=x_2=x_3=x_4=\frac{1}{2+\lambda}\]
6.\\
显然,由于每个矩阵都行等价于一个行简化阶梯矩阵,因此若$m<n$,则主元数量少于变量,因此要么有无穷多组解,要么无解。又因为必有零解,因此有无穷多个非零解。
\end{document}
































