\documentclass[utf8]{ctexart}
\usepackage{graphicx}
\usepackage{amsmath}
\usepackage{amssymb}
\usepackage{siunitx}
\makeatletter
\newcommand{\rmnum}[1]{\romannumeral #1}
\newcommand{\Rmnum}[1]{\expandafter\@slowromancap\romannumeral #1@}
\makeatother
\title{统计力学作业4}
\author{郑子诺,物理41}
\date{\today}
\begin{document}
\maketitle
\noindent3.20\\
根据朗道自由能公式我们有
\[F(T)=\begin{cases}
	F_0(T)-\dfrac{a_0^2(T_C-T)^2}{4b}&T<T_C\\[8pt]
	F_0(T)&T>T_C
\end{cases}\]
根据$S=-\dfrac{\mathrm{d}F}{\mathrm{d}T}$我们有
\[S(T)=\begin{cases}
	-\dfrac{\mathrm{d}F_0}{\mathrm{d}T}+\dfrac{a_0^2(T-T_C)}{2b}&T<T_C\\[8pt]
	-\dfrac{\mathrm{d}F_0}{\mathrm{d}T}&T>T_C
\end{cases}\]
显然连续。\\
选做题\\
求导为$0$算出极值点
\[m^2=\frac{\pm\sqrt{\frac{16}{9}c^2-6bd}-\frac{4}{3}c}{3d},0\]
求二阶导易知,只有取正号才是稳定平衡。观察函数图样随$b$的变化,发现$b<0$时,$0$为不稳定平衡点,除此之外还有两个稳定平衡点。当$b>0$时,又出现两个不稳定平衡点,$0$变成稳定平衡点。当$b$越来越大时,两个稳定平衡点逐渐上升,超过$0$,直至最后消失。由此观之,一级相变将发生在两个稳定平衡点等于$0$的附近,此时$b$为
\[-\frac{1}{3}c-\frac{1}{2}dm^2=0\rightarrow b_0=\frac{2c^2}{9d}\]
$m$为
\[m=\pm\sqrt{-\frac{2c}{3d}}\]
在$b_0$附近展开$b$为$b_0+b_1(T-T_C)$,只保留一阶项,再根据$S=-\dfrac{\mathrm{d}G}{\mathrm{d}T}$得
\[L=T\Delta S=T\left|\frac{b_1c}{3d}\right|\]
其中$m$对$T$的依赖正好被消掉。\\
作业题目1\\
(1)直接积分得$C=\dfrac{4}{a^2b^2}$,因此
\[p(x,y)=\frac{4xy}{a^2b^2}\]
(2)对$x$积分得
\[p(y)=\frac{2}{b^2}y\]
因此我们有
\[p(x|y)=\frac{p(x,y)}{p(y)}=\frac{2}{a^2}x\]
作业题目2\\
(1)
\[\bar{X}=\int_{0}^{+\infty}axe^{-ax}\mathrm{d}x=\frac{1}{a}\]
(2)
\[\bar{X^2}=\int_{0}^{+\infty}ax^2e^{-ax}\mathrm{d}x=\frac{2}{a^2}\]
\[\Delta X=\sqrt{\bar{X^2}-\bar{X}^2}=\frac{1}{a}\]
(3)
\[\delta X=1\]
\end{document}