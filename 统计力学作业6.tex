\documentclass[utf8]{ctexart}
\usepackage{graphicx}
\usepackage{amsmath}
\usepackage{amssymb}
\usepackage{siunitx}
\makeatletter
\newcommand{\rmnum}[1]{\romannumeral #1}
\newcommand{\Rmnum}[1]{\expandafter\@slowromancap\romannumeral #1@}
\makeatother
\title{统计力学作业6}
\author{郑子诺,物理41}
\date{\today}
\begin{document}
\maketitle
\noindent6.2\\
根据薛定谔方程,我们有
\[\frac{\hbar^2k^2}{2m}=E\]
由边界条件得
\[k=\frac{n\pi}{L},n\in\mathbb{Z}\]
因此波矢空间内每$\dfrac{\pi}{L}$有一个态。因而能量在$\epsilon$内的态数为
\[\Omega(\epsilon)=\frac{L}{\pi}\sqrt{\frac{2m\epsilon}{\hbar^2}}\]
求导得
\[D(\epsilon)\mathrm{d}\epsilon=\frac{2L}{h}\sqrt{\frac{m}{2\epsilon}}\mathrm{d}\epsilon\]
6.3\\
类似地,此时能量满足
\[k_1^2+k_2^2=\frac{2m\epsilon}{\hbar^2}\]
而$k_1,k_2$满足
\[k_1=\frac{n_1\pi}{L},k_2=\frac{n_2\pi}{L},n_1,n_2\in\mathbb{Z}\]
因此波矢空间每$\dfrac{\pi^2}{L^2}$有一个态。因而能量在$\epsilon$内的态数为
\[\Omega(\epsilon)=\frac{L^2}{\pi^2}\frac{\pi}{4}\frac{2m\epsilon}{\hbar^2}\]
求导得
\[D(\epsilon)\mathrm{d}\epsilon=\frac{2\pi L^2}{h^2}m\mathrm{d}\epsilon\]
6.4\\
与前同理,唯一改变的是能量与波矢的关系,此时为
\[\epsilon=c\hbar\sqrt{k_1^2+k_2^2+k_3^2}\]
因此能量在$\epsilon$内的态数为
\[\Omega(\epsilon)=\frac{L^3}{8\pi^3}\frac{4\pi}{3}\frac{\epsilon^3}{c^3\hbar^3}\]
求导得
\[D(\epsilon)\mathrm{d}\epsilon=\frac{4\pi L^3\epsilon^2}{h^3c^3}\mathrm{d}\epsilon\]
7.1\\
根据$V=L^3$,我们有
\[\frac{\partial\epsilon}{\partial V}=-\frac{2}{3}\frac{1}{2m}\frac{h^2}{V^\frac{5}{3}}(n_x^2+n_y^2+n_z^2)=-\frac{2}{3}\frac{\epsilon}{V}\]
所以
\[P=\sum-a_i\frac{\partial\epsilon_i}{\partial V}=\frac{2}{3}\frac{1}{V}\sum a_i\epsilon_i=\frac{2}{3}\frac{U}{V}\]
7.2\\
类似上题,我们有
\[\frac{\partial\epsilon}{\partial V}=-\frac{1}{3}c\frac{h}{V^\frac{4}{3}}(n_x^2+n_y^2+n_z^2)^\frac{1}{2}=-\frac{1}{3}\frac{\epsilon}{V}\]
所以
\[P=\sum-a_i\frac{\partial\epsilon_i}{\partial V}=\frac{1}{3}\frac{1}{V}\sum a_i\epsilon_i=\frac{1}{3}\frac{U}{V}\]
7.3\\
我们有
\[Z_l^*=\sum\omega_ie^{-\alpha-\beta\epsilon_i^*}=\sum\omega_ie^{-\alpha-\beta(\epsilon_i+\Delta)}=e^{-\beta\Delta}Z_l\]
此时热力学函数分别为
\[U^*=-N\frac{\partial\ln Z_l^*}{\partial\beta}=U+N\Delta\]
\[Y_k^*=-\frac{N}{\beta}\frac{\partial\ln Z_l^*}{\partial y_k}=Y_k\]
\[S^*=Nk\left(\ln Z_l^*-\beta\frac{\partial\ln Z_l^*}{\partial\beta}\right)=S\]
\[F^*=U^*-TS^*=F+N\Delta\]
7.6\\
(1)当存在$n$个缺陷和填隙原子时,我们有$C_N^n$种可能的缺陷位置,以及$C_N^n$种可能的填隙位置,根据乘法原理我们有
\[S=k\ln\Omega=k\ln(C_N^n)^2=2k\ln\frac{N!}{(N-n)!n!}\]
(2)我们有
\[F=nu-TS\approx nu-2kT\ln N!+2kT(n\ln n-n)+2kT[(N-n)ln(N-n)-(N-n)]\]
对$n$求导为零得
\[u+2kT\ln n-2kT\ln(N-n)=0\]
所以
\[\frac{n}{N}\approx\frac{n}{N-n}=e^{-\frac{u}{2kT}}\]
\[n=Ne^{-\frac{u}{2kT}}\]
\end{document}



















