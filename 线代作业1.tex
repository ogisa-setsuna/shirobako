\documentclass[utf8]{ctexart}
\usepackage{graphicx}
\usepackage{amsmath}
\title{线性代数第一次作业}
\author{郑子诺,物理41}
\date{\today}
\begin{document}
\maketitle
\noindent Q1:\\
(a):是。\\因为由$x_1+y_1+z_1=0$,$x_2+y_2+z_2=0$可得$$(ax_1+bx_2)+(ay_1+by_2)+(az_1+bz_2)=0$$对加法和数乘封闭,因此是线性子空间。\\
(b):不是。\\因为由$x_1+y_1+z_1=1$,$x_2+y_2+z_2=1$可得$$(ax_1+bx_2)+(ay_1+by_2)+(az_1+bz_2)=a+b$$对加法和数乘不封闭,因此不是线性子空间。\\
(c):不是。\\因为由$x_1^2+y_1^2-z_1^2=0$,$x_2^2+y_2^2-z_2^2=0$可得$$(ax_1^2+bx_2^2)+(ay_1^2+by_2^2)-(az_1^2+bz_2^2)=0\neq (ax_1+bx_2)^2+(ay_1+by_2)^2-(az_1+bz_2)^2$$对加法和数乘不封闭,因此不是线性子空间。\\
(d):是。\\因为由$x_1=2y_1,z_1=3y_1$,$x_2=2y_2,z_2=3y_2$可得$$ax_1+bx_2=2(ay_1+by_2),az_1+bz_2=3(ay_1+by_2)$$对加法和数乘封闭,因此是线性子空间。\\
(e):是。\\显然当数乘负数时该子空间不封闭,因此不是线性子空间。\\
Q2:\\
(a):构成。\\因为由$f(0)=0$,$g(0)=0$可得$$af(0)+bg(0)=0$$对加法和数乘封闭,因此是线性子空间。\\
(b):构成。\\因为由$f(x)=ax+b$,$g(x)=cx+d$可得$$Af(x)+Bg(x)=(Aa+Bc)x+(Ab+Bd)$$对加法和数乘封闭,因此是线性子空间。\\
Q3:\\
(a):$\mathbf{v_1},\mathbf{v_2},\mathbf{v_3}$的行列式为
$$
\begin{vmatrix}
    1  &0  &-1\\
	2  &1  &0\\
	3  &2  &1\\
\end{vmatrix}
$$
值为$0$,因此线性相关,有$$\mathbf{v_1}-2\mathbf{v_2}+\mathbf{v_3}=0$$\\
(b):可以。$$\mathbf{u_1}=\mathbf{v_1}+3\mathbf{v_2}$$\\
(c):不可以。由于$\mathbf{v_1},\mathbf{v2},\mathbf{v_3}$线性相关,$\mathbf{v_3}$可表示为$\mathbf{v_1},\mathbf{v_2}$的线性组合,因此实际上是在求$\mathbf{u_2}$关于$\mathbf{v_1},\mathbf{v_2}$的线性组合。根据$\mathbf{u_2}$的$x$分量可知,$\mathbf{v_1}$的系数为$4$,又由$\mathbf{u_2}$的$y$分量可知,$\mathbf{v_2}$的系数为$-1$,此时不符合$\mathbf{u_2}$的$z$分量。因此不可以表示为$\mathbf{v_1},\mathbf{v2},\mathbf{v_3}$的线性组合。\\
Q4:\\
(a):显然,$a+bx+cx^2=0$当且仅当$a=b=c=0$,因此$1,x,x^2$线性无关。\\
(b):由于$1,x,x^2$线性无关,且可线性表示一切次数不超过$2$的多项式,因此将$1,x,x^2$作为基矢。\\此时$1-x,1+x,x^2$的行列式为
$$
\begin{vmatrix}
	1  &1  &0\\
	-1  &1  &0\\
	0  &0  &1\\
\end{vmatrix}
$$
值为$2$,因此线性无关。\\
(c):$1,x,x^2,1+2x+3x^2$的个数超过线性空间维数$3$,因此一定线性相关。\\
(d):因为$1-x,1+x,x^2$线性无关,因此可选作基矢,于是$q(x)$可表示为$$q(x)=\frac{7}{2}p_1(x)+\frac{1}{2}p_2(x)+2p_3(x)$$\\
(e):理由同上,有$$r(x)=-\frac{1}{2}p_1(x)-\frac{1}{2}p_2(x)+p_3$$\\
Q5:\\
因为存在不是全为$0$的数组$a_1,a_2,...,a_n$使得
$$a_1\mathbf{v_1}+a_2\mathbf{v_2}+...+a_n\mathbf{v_n}=0$$
因此取$a_{n+1}=0$,则存在不是全为$0$的数组$a_1,a_2,...,a_{n+1}$使得
$$a_1\mathbf{v_1}+a_2\mathbf{v_2}+...+a_{n+1}\mathbf{v_{n+1}}=0$$\\
Q.E.D.\\
Q6:\\
若$\mathbf{v_1},\mathbf{v_2},\mathbf{v_3}$线性相关,则存在不是全为$0$的数组$a_1,a_2,a_3$使得
$$a_1\mathbf{v_1}+a_2\mathbf{v_2}+a_3\mathbf{v_3}=0$$
则必有一个系数不为$0$。设该系数为$a_3$,则
$$\mathbf{v_3}=-\frac{a_1}{a_3}\mathbf{v_1}-\frac{a_2}{a_3}\mathbf{v_2}$$
与条件矛盾。\\
若该系数为$a_1$,则
$$\mathbf{v_1}=-\frac{a_2}{a_1}\mathbf{v_2}-\frac{a_3}{a_1}\mathbf{v_3}$$
由题设条件可知$a_3\neq 0$,于是回归到先前讨论。$a_2$同理。\\
于是可得假设不成立。\\
Q.E.D.\\
Q7:\\
存在不是全为$0$的数组$a_1,a_2,...,a_n,b$使得
$$a_1\mathbf{v_1}+a_2\mathbf{v_2}+...+a_n\mathbf{v_n}+b\mathbf{u}=0$$
若$b\neq 0$,则
$$\mathbf{u}=-\frac{a_1}{b}\mathbf{v_1}-\frac{a_2}{b}\mathbf{v_2}-...-\frac{a_n}{b}\mathbf{v_n}$$
结论成立。\\
若$b=0$,则由题设条件知$a_1=a_2=...=a_n=0$,与线性相关矛盾。\\
所以$b\neq 0$,因而结论成立。\\
Q.E.D.
\end{document}





















