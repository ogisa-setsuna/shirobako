\documentclass[utf8]{ctexart}
\usepackage{graphicx}
\usepackage{amsmath}
\usepackage{braket}
\usepackage{amssymb}
\makeatletter
\newcommand{\rmnum}[1]{\romannumeral #1}
\newcommand{\Rmnum}[1]{\expandafter\@slowromancap\romannumeral #1@}
\makeatother
\title{费曼3作业5}
\author{郑子诺,物理41}
\date{\today}
\begin{document}
\maketitle
\noindent75.2\\
\[C'=R_{y}(\frac{\pi}{4})C\]
\[\begin{bmatrix}
	\cos\frac{\pi}{8}&\sin\frac{\pi}{8}\\
	-\sin\frac{\pi}{8}&\cos\frac{\pi}{8}
\end{bmatrix}
\begin{bmatrix}
	1\\
	0
\end{bmatrix}
=\begin{bmatrix}
	\cos\frac{\pi}{8}\\
	-\sin\frac{\pi}{8}
\end{bmatrix}\]
\[\therefore\ket{\phi(t)}=\cos\frac{\pi}{8}e^{i\frac{\mu B}{\hbar}t}\ket{+'}-\sin\frac{\pi}{8}e^{-i\frac{\mu B}{\hbar}t}\ket{-'}\]
\begin{align*}
	P_{+x}&=|\braket{+x|+'}+\braket{+x|-'}|^2\\
	&=|\cos^2\frac{\pi}{8}e^{i\frac{\mu B}{\hbar}t}-\sin^2\frac{\pi}{8}e^{-i\frac{\mu B}{\hbar}t}|^2\\
	&=|\cos\frac{\pi}{4}\cos(\frac{\mu B}{\hbar}t)+i\sin(\frac{\mu B}{\hbar}t)|^2\\
	&=\frac{1}{2}(1+\sin^2(\frac{\mu B}{\hbar}t))
\end{align*}
\[\because C_y=R_z(\frac{\pi}{2})C_x\rightarrow C_{+y}=e^{i\frac{\pi}{4}}C_{+x}\]
\[\therefore P_{+y}=P_{+x}=\frac{1}{2}(1+\sin^2(\frac{\mu B}{\hbar}t))\]
75.3\\
(a)According to the condition, we have
\begin{align*}
	\frac{dC_+}{dt}&=i\frac{A}{\hbar}C_-\\
	\frac{dC_-}{dt}&=i\frac{A}{\hbar}C_+
\end{align*}
Thus we have
\[\frac{d^2C_+}{dt^2}+\frac{A^2}{\hbar^2}C_+=0\]
and the initial condition
\[C_+(0)=1,\frac{dC_+}{dt}(0)=0\]
\[\therefore P_+=|C_+(t)|^2=\cos^2(\frac{A}{\hbar}t)\]
(b)We can find the eigenvectors of $H$
\[\begin{vmatrix}
	-\lambda&-A\\
	-A&-\lambda
\end{vmatrix}=0\rightarrow\lambda=A,-A\]
\[\alpha_1=\begin{bmatrix}
	\frac{1}{\sqrt{2}}\\
	\frac{1}{\sqrt{2}}
\end{bmatrix},\alpha_2=\begin{bmatrix}
\frac{1}{\sqrt{2}}\\
-\frac{1}{\sqrt{2}}
\end{bmatrix}\]
Thus we have the stationary states
\[\ket{\phi_1}=\frac{1}{\sqrt{2}}\ket{+}+\frac{1}{\sqrt{2}}\ket{-}\]
\[\ket{\phi_2}=\frac{1}{\sqrt{2}}\ket{+}-\frac{1}{\sqrt{2}}\ket{-}\]
and the new Hamiltonian matrix
\[H'=\begin{bmatrix}
	A&0\\
	0&-A
\end{bmatrix}\]
Hence the energies of the two states are
\[E_1=A,E_2=-A\]
(c)For an arbitrary direction,we have
\[\ket{+}=\cos\frac{\theta}{2}e^{i\frac{\phi}{2}}\ket{+'}-i\sin\frac{\theta}{2}e^{i\frac{\phi}{2}}\ket{-'}\]
\[\ket{-}=\sin\frac{\theta}{2}e^{-i\frac{\phi}{2}}\ket{+'}+i\cos\frac{\theta}{2}e^{-i\frac{\phi}{2}}\ket{-'}\]
which can be acquired by multiple $R_z(\phi-\frac{\pi}{2})$ and $R_x(\theta)$.\\
Thus we have
\[C_{+'}=\cos(\frac{A}{\hbar}t)\cos\frac{\theta}{2}e^{i\frac{\phi}{2}}+i\sin(\frac{A}{\hbar}t)\sin\frac{\theta}{2}e^{-i\frac{\phi}{2}}\]
\[P_{+'}=|C_{+'}|^2=\sin^2\frac{\theta}{2}+\cos^2(\frac{A}{\hbar}t)\cos\theta+\sin(\frac{A}{\hbar}t)\cos(\frac{A}{\hbar}t)\sin\theta\sin\phi=1\]
\[\cos2(\frac{A}{\hbar}t)\cos\theta+\sin2(\frac{A}{\hbar}t)\sin\theta\sin\phi=1\]
\[\sqrt{\cos^2\theta+\sin^2\theta\sin^2\phi}\cos(2\frac{A}{\hbar}t-\delta)=1,\tan\delta=\frac{\sin\theta\sin\phi}{\cos\theta}\]
\[\therefore\phi=\frac{\pi}{2},\theta=2\frac{A}{\hbar}t+2k\pi\]
Another method:
\[\cos\frac{\theta}{2}e^{-i\frac{\phi}{2}}=c\cos\frac{A}{\hbar}t\]
\[\sin\frac{\theta}{2}e^{i\frac{\phi}{2}}=ci\sin\frac{A}{\hbar}t\]
\[\therefore\phi=\frac{\pi}{2},\theta=2\frac{A}{\hbar}t+2k\pi\]
(d)An equipment with $x$-axis uniform magnetic field.\\
76.1\\
(a)The probability of transmission is
\[P'(T)=4\pi^2(\frac{\mu^2}{4\pi\epsilon_0\hbar^2c})\mathcal{J}(\omega_0)T\]
Thus the probability per unit time is
\[P(\Rmnum1\rightarrow\Rmnum2)=4\pi^2(\frac{\mu^2}{4\pi\epsilon_0\hbar^2c})\mathcal{J}(\omega_0)\]
(b)Because the equations are symmetric, we immediately have
\[P(\Rmnum2\rightarrow\Rmnum1)=4\pi^2(\frac{\mu^2}{4\pi\epsilon_0\hbar^2c})\mathcal{J}(\omega_0)\]
(c)We have
\[B_{\Rmnum1\rightarrow\Rmnum2}=B_{\Rmnum2\rightarrow\Rmnum1}=4\pi^2(\frac{\mu^2}{4\pi\epsilon_0\hbar^2c})\]
(d)
\[A_{\Rmnum{1}\rightarrow\Rmnum{2}}=4\pi^2(\frac{\mu^2}{4\pi\epsilon_0\hbar^2c})\frac{\hbar\omega_0^3}{\pi^2c^2}\]
\end{document}



























