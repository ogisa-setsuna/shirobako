\documentclass[utf8]{ctexart}
\usepackage{graphicx}
\usepackage{amsmath}
\usepackage{amssymb}
\makeatletter
\newcommand{\rmnum}[1]{\romannumeral #1}
\newcommand{\Rmnum}[1]{\expandafter\@slowromancap\romannumeral #1@}
\makeatother
\title{高微2作业3}
\author{郑子诺,物理41}
\date{\today}
\begin{document}
\maketitle
\noindent1.\\
(1)首先$|f(x)-g(x)|$显然也是连续函数,由于连续函数在闭区间上可以取到最大值,该度量是良定义的。\\
其次,正定性是显然的,因为$d(f,g)$肯定大于等于$0$,且若其为$0$则有
\[0\le|f(x)-g(x)|\le d(f,g)=0\]
因此其等于$0$当且仅当$f=g$。\\
对称性也是显然的,而三角不等式可由普通的三角不等式得出。首先
\[|f(x)-h(x)|\le|f(x)-g(x)|+|g(x)-h(x)|\]
设左边在$x_0$处取到最大值,我们有
\[d(f,h)=|f(x_0)-h(x_0)|\le|f(x_0)-g(x_0)|+|g(x_0)-h(x_0)|\le d(f,g)+d(g,h)\]
(2)根据定义,我们知道对于任一$x\in I$,当$n,m\ge N$时我们有
\[|f_n(x)-f_m(x)|\le d(f_n,f_m)<\epsilon\]
根据实数的柯西收敛定理知每点都有极限$f(x)$。首先证$f$为连续函数,为此,先对上式取极限
\[\lim_{m\rightarrow+\infty}|f_n(x)-f_m(x)|=|f_n(x)-f(x)|\le\epsilon\]
由于$f_n$为连续函数,对于任一$\epsilon$,存在$\delta>0$使得$|x-x_0|<\delta$时有
\[|f_n(x)-f_n(x_0)|<\epsilon\]
因此我们有
\[|f(x)-f(x_0)|\le |f(x)-f_n(x)|+|f_n(x)-f_n(x_0)|+|f_n(x_0)-f(x_0)|<3\epsilon\]
因此$f$为连续函数。最后由$n\ge N$时$|f_n(x)-f(x)|\le\epsilon$得
\[d(f_n,f)\le\epsilon<2\epsilon\]
因此有
\[\lim_{n\rightarrow+\infty}f_n=f\]
2.\\
取任意实数$t$,根据积分不等式我们有
\[\int_{a}^{b}(f-tg)^2\ge0\]
展开得
\[\int_{a}^{b}f^2+t^2\int_{a}^{b}g^2-2t\int_{a}^{b}fg\ge0\]
由于对于所有$t$成立,该二次函数判别式需小于等于$0$,因此我们有
\[4\left(\int_{a}^{b}fg\right)-4\left(\int_{a}^{b}f^2\right)\left(\int_{a}^{b}g^2\right)\le0\]
于是
\[\left(\int_{a}^{b}fg\right)\le\left(\int_{a}^{b}f^2\right)\left(\int_{a}^{b}g^2\right)\]
3.\\
由于$U$是开集,任一$x\in U$为$U$的内点,因此存在开球邻域$B_{x,r}\subseteq U$。于是我们有
\[U\subseteq\bigcup_{x\in U}B_{x,r}\subseteq U\]
因此$U$可以表示成一族开球邻域的并。反之,对于任一$x\in U$存在一开球使得$x\in B_{x_0,r_0}$,取$r<r_0-d(x,x_0)$,我们有
\[B_{x,r}\subseteq B_{x_0,r_0}\subseteq\bigcup B_{x_i,r_i}=U\]
因此$U$是开集。\\
4.\\
(1)由于
\[1-\cos(xy)=2\sin^2\frac{xy}{2}\le\frac{x^2y^2}{2}\]
令$x=r\cos\theta,y=r\sin\theta$,我们有
\[0\le\frac{1-\cos(xy)}{x^2+y^2}\le\frac{x^2y^2}{2(x^2+y^2)}=\frac{r^2}{2}\sin^2\theta\cos^2\theta\le r^2\]
根据夹逼定理得该函数极限为$0$。\\
(2)若存在极限$L$,取$x=t,y=kt$,显然满足复合函数极限定理条件\Rmnum{1},我们有
\[L=\lim_{t\rightarrow0^+}\frac{-3kt^3+k^3t^3-4kt^2}{(1+k^2)t^2}=-\frac{4k}{1+k^2}\]
显然为$k$的函数,因此矛盾。所以极限不存在。\\
(3)首先证明函数$\dfrac{(1+x)^{\frac{1}{x}-e}}{x}$在$x=0$处存在极限。由于上下趋于$0$,利用洛必达得
\begin{align*}
	&\lim_{x\rightarrow0}\frac{(1+x)^\frac{1}{x}-e}{x}\\
	&=\lim_{x\rightarrow0}\left(-\frac{1}{x^2}\ln(1+x)+\frac{1}{(1+x)x}\right)(1+x)^\frac{1}{x}\\
	&=\lim_{x\rightarrow0}\left(-\frac{1}{1+x}+\frac{1}{2}+\frac{o(x^3)}{x^2}\right)(1+x)^\frac{1}{x}\\
	&=-\frac{e}{2}
\end{align*}
再根据复合函数极限定理,显然满足条件\Rmnum{1},得到原函数极限为$-\dfrac{e}{2}$。\\
(4)令$x=r\cos\theta,y=r\sin\theta$,我们有
\[\frac{ax+by}{(x^2+y^2)^\alpha}=r^{1-2\alpha}(a\cos\theta+b\sin\theta)\]
当$\alpha<\dfrac{1}{2}$时,我们有
\[-\sqrt{a^2+b^2}r^{1-2\alpha}\le r^{1-2\alpha}(a\cos\theta+b\sin\theta)\le\sqrt{a^2+b^2}r^{1-2\alpha}\]
因此极限为$0$。当$\alpha>\dfrac{1}{2}$时,若存在极限$L$,取$x=t,y=kt$,根据复合函数定理满足条件\Rmnum{1},我们有
\[L=\lim_{t\rightarrow0^+}\frac{(a+bk)t^{1-2\alpha}}{(1+k^2)^\alpha}\]
若$a=b=0$,极限为$0$,否则取$k$使分子不为$0$,必有极限趋于无穷。若$\alpha=\dfrac{1}{2}$,要求极限存在且等于$0$,至少要求
\[\frac{a+bk}{\sqrt{1+k^2}}=0\]
对于所有$k$成立。因此$a=b=0$。综上所述,我们要求$\alpha<\dfrac{1}{2}$或$a=b=0$。\\
5.\\
根据三角不等式我们有
\[d(x_0,x)-d(x_0,x')\le d(x',x)\]
\[d(x_0,x')-d(x_0,x)\le d(x,x')=d(x',x)\]
因此
\[|d(x_0,x)-d(x_0,x')|\le d(x',x)\]
对于任一$\epsilon>0$,取$\delta<\epsilon$,对于$d(x,x')<\delta$我们总有
\[|d(x_0,x)-d(x_0,x')|\le d(x',x)<\epsilon\]
因此该函数连续。\\
6.\\
由于$h(x)$可以写作
\[h(x)=\frac{1}{2}(f(x)+g(x)-|f(x)-g(x)|)\]
且对连续函数求和求绝对值后仍然连续,因此$h(x)$为连续函数。
\end{document}















