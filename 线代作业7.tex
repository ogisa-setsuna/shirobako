\documentclass[utf8]{ctexart}
\usepackage{graphicx}
\usepackage{amsmath}
\usepackage{amssymb}
\title{线代作业7}
\author{郑子诺,物理41}
\date{\today}
\begin{document}
\maketitle
1.\\
(1)\[
\begin{bmatrix}
	-3&1&2\\
	1&0&-1\\
	3&-1&-1
\end{bmatrix}\]
(2)\[\begin{bmatrix}
	-\dfrac{1}{4}&-\dfrac{1}{28}&\dfrac{25}{28}&\dfrac{5}{28}\\[8pt]
	0&\dfrac{1}{7}&-\dfrac{4}{7}&\dfrac{2}{7}\\[8pt]
	-\dfrac{1}{4}&\dfrac{11}{28}&\dfrac{5}{28}&\dfrac{1}{28}\\[8pt]
	\dfrac{1}{2}&-\dfrac{5}{14}&-\dfrac{1}{14}&-\dfrac{3}{14}
\end{bmatrix}\]
(3)\[\begin{bmatrix}
	1&-1&0&0&\cdots&0\\
	0&1&-1&0&\cdots&0\\
	0&0&1&-1&\cdots&0\\
	\vdots&\vdots&\vdots&\vdots&\ddots&\vdots\\
	0&0&\cdots&0&1&-1\\
	0&0&\cdots&0&0&1
\end{bmatrix}\]
(4)经过艰苦卓绝的一系列行变换我们终于得到了
\[\begin{bmatrix}
	-\dfrac{n}{n+1}&1-\dfrac{2n}{n+1}&2-\dfrac{3n}{n+1}&\cdots&n-1-\dfrac{n^2}{n+1}\\[8pt]
	-\dfrac{n-1}{n+1}&-\dfrac{2(n-1)}{n+1}&1-\dfrac{3(n-1)}{n+1}&\cdots&n-2-\dfrac{n(n-1)}{n+1}\\[8pt]
	\vdots&\vdots&\vdots&\ddots&\vdots\\[8pt]
	-\dfrac{1}{n+1}&-\dfrac{2}{n+1}&-\dfrac{3}{n+1}&\cdots&-\dfrac{n}{n+1}
\end{bmatrix}\]
2.\\
\[(I_n-A)X=0\]
只需证明该方程有唯一解。
\[X=AX\]
\[AX=A^2X\]
以此类推得到
\[A=AX=A^2X=\cdots=A^NX=0\]
因此$I_n-A$为可逆矩阵。\\
3.\\
让我们求伴随矩阵
\[(adj\,A)_{ij}=(-1)^{i+j}A_{ji}\]
令$i,j$处于下三角,显然$j,i$处于上三角,因此$A_{ji}$为一个上三角元的余子式,显然也是一个上三角矩阵,并且其对角线至少含有一个$0$,因此行列式为$0$。于是伴随矩阵也是上三角矩阵。\\
4.\\
显然,初等行变换并不会导致行列式变为$0$。我们对$A$做初等行变换,相当于对每一个行列子式做初等行变换,因而并不会导致任何行列子式变为$0$。初等列变换同理。于是我们可以通过一系列初等行变换将$A$变为简化阶梯矩阵,再通过初等列变换使其变为标准型
\[\begin{bmatrix}
	I_r&0\\
	0&0
\end{bmatrix}\]
很显然此时$p(A)=rank(A)$。\\
5.\\
显然,根据之前作业所证过的定理,初等行变换等价于左乘一个初等矩阵,初等列变换等价于右乘一个初等矩阵,而且显然初等矩阵是可逆的,因为初等行、列变换都是可逆的。我们显然可以通过一系列初等行、列变换使得$A$化为标准型,而可逆矩阵的乘积仍然可逆,于是我们有
\[A=P\begin{bmatrix}
	I_r&0\\
	0&0
\end{bmatrix}Q\]
\end{document}






















