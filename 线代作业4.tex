\documentclass[utf8]{ctexart}
\usepackage{graphicx}
\usepackage{amsmath}
\usepackage{amsthm}
\title{线代作业4}
\author{郑子诺,物理41}
\date{\today}
\begin{document}
\maketitle
\noindent1.\\
(1)
系数矩阵通过初等行变换可化为:
\[\begin{bmatrix}
	1&0&-1&-1&-5\\
	0&1&2&2&6\\
	0&0&0&0&0\\
	0&0&0&0&0
\end{bmatrix}\]
所以通解为:
\[\begin{cases}
	x_1=x_3+x_4+x_5\\
	x_2=-2x_3-2x_4-6x_5
\end{cases}\]
(2)
系数矩阵通过初等行变换可化为:
\[\begin{bmatrix}
	1&0&0&0&-\frac{1}{3}\\
	0&1&0&0&\frac{1}{3}\\
	0&0&1&0&0\\
	0&0&0&1&-\frac{1}{3}
\end{bmatrix}\]
所以通解为:
\[\begin{cases}
	x_1=\frac{1}{3}x_5\\
	x_2=-\frac{1}{3}x_5\\
	x_3=0\\
	x_4=\frac{1}{3}x_5
\end{cases}\]
2.\\
(1)增广矩阵通过初等行变换可化为:
\[\begin{bmatrix}
	1&0&0&0&-8\\
	0&1&0&-1&3\\
	0&0&1&-2&6\\
	0&0&0&0&0
\end{bmatrix}\]
所以通解为:
\[\begin{cases}
	x_1=-8\\
	x_2=x_4+3\\
	x_3=2x_4+6
\end{cases}\]
(2)(1)增广矩阵通过初等行变换可化为:
\[\begin{bmatrix}
	1&0&0&0\\
	0&1&0&2\\
	0&0&1&1\\
	0&0&0&-1
\end{bmatrix}\]
所以无解。
3.\\
显然,齐次线性方程组可以写为:
\[\begin{bmatrix}
	\alpha_1&\alpha_2&\dots&\alpha_n
\end{bmatrix}
\begin{bmatrix}
	c_1\\
	c_2\\
	\vdots\\
	c_n
\end{bmatrix}
=\begin{bmatrix}
	0\\
	0\\
	\vdots\\
	0
\end{bmatrix}\]
其中$\alpha_i$为矩阵列向量。容易发现,若有列秩为$r$,则根据线性无关性,作为基矢的$r$个列向量的系数$c_i$可由剩下的$n-r$个系数表示,因此$n-r$即解空间维数。显然,初等行变换不改变矩阵的解空间,因此初等行变换不改变列秩。\\
4.\\
由3知,选定$n-1$个基矢,其系数可由剩下一个系数表示,由于方程齐次,必为倍数关系,因此两两解之间差固定倍数。\\
5.\\
由于秩为$1$,因此不失一般性设$\alpha_1$为基矢,根据前两问讨论我们有:
\begin{align}
	A=\begin{bmatrix}
		\alpha_1&\alpha_2&\dots&\alpha_n
	\end{bmatrix}
	&=\begin{bmatrix}
		\alpha_1&\alpha_1&\dots&\alpha_1
	\end{bmatrix}
	\begin{bmatrix}
		b_1\\
		b_2\\
		\vdots\\
		b_n
	\end{bmatrix}\notag\\
	&=\begin{bmatrix}
		a_1\\
		a_2\\
		\vdots\\
		a_m
	\end{bmatrix}
	\begin{bmatrix}
		1&1&\dots&1
	\end{bmatrix}
	\begin{bmatrix}
		b_1\\
		b_2\\
		\vdots\\
		b_n
	\end{bmatrix}\notag\\
	&=\begin{bmatrix}
		a_1\\
		a_2\\
		\vdots\\
		a_m
	\end{bmatrix}
	\begin{bmatrix}
		b_1&b_2&\dots&b_n
	\end{bmatrix}\notag
\end{align}
\end{document}











