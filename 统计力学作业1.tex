\documentclass[utf8]{ctexart}
\usepackage{graphicx}
\usepackage{amsmath}
\usepackage{amssymb}
\usepackage{siunitx}
\title{统计力学作业1}
\author{郑子诺,物理41}
\date{\today}
\begin{document}
\maketitle
\noindent1.3\\
\[-\frac{1}{V}\left(\frac{\partial V}{\partial p}\right)_T=\kappa_T\]
\[V_1=V_0\exp(-\kappa_Tp)\approx V_0(1-\kappa_Tp)\]
\[\frac{1}{V}\left(\frac{\partial V}{\partial T}\right)_p=\alpha\]
\begin{align*}
	V(T,p)&=V_1(T_0,p)\exp[\alpha(T-T_0)]\\
	&\approx V_0(T_0,0)[1+\alpha(T-T_0)-\kappa_Tp]
\end{align*}
1.5\\
由于
\[\left(\frac{\partial T}{\partial\mathcal{F}}\right)_L\left(\frac{\partial\mathcal{F}}{\partial L}\right)_T\left(\frac{\partial L}{\partial T}\right)_\mathcal{F}=-1\]
因此
\[\left(\frac{\partial\mathcal{F}}{\partial T}\right)_L=-EA\alpha\]
\[\Delta\mathcal{F}=-EA\alpha(T_2-T_1)\]
1.7\\
跟踪进入箱子中的气体,显然周围其他气体将其压入箱子所做的功为$p_0V_0$,而箱子内为真空,且忽略气体间热传导,因此我们有
\[U-U_0=p_0V_0\]
若为理想气体,则有
\[\frac{5}{2}nR(T-T_0)=p_0V_0=nRT_0\]
\[nRT=p_0V\]
所以
\[T=\frac{7}{5}T_0\]
\[V=\frac{7}{5}V_0\]
其中$T_0$为大气温度。\\
1.11\\
根据绝热方程
\[pV^\gamma=C\rightarrow\frac{T^\gamma}{p^{\gamma-1}}=C\]
我们有
\[\gamma\frac{\mathrm{d}T}{T}=(\gamma-1)\frac{\mathrm{d}p}{p}\]
同时我们有
\[\mathrm{d}p=-\rho g\mathrm{d}z\]
\[\rho=\frac{\mu p}{RT}\]
所以
\[\frac{\mathrm{d}T}{\mathrm{d}z}=-\frac{\gamma-1}{\gamma}\frac{\mu g}{R}\]
代入数值得
\[\frac{\mathrm{d}T}{\mathrm{d}z}=\qty[per-mode=symbol]{-9.77e-3}{\kelvin\per\meter}\]
\end{document}