\documentclass[utf8]{ctexart}
\usepackage{graphicx}
\usepackage{amsmath}
\usepackage{amssymb}
\title{高微2作业1}
\author{郑子诺,物理41}
\date{\today}
\begin{document}
\maketitle
\noindent1.\\
(1)由于
\[|\sum_{n=1}^n\sin n\theta|=|\frac{\sin\frac{n\theta}{2}\theta\sin\frac{(n+1)\theta}{2}}{\sin\frac{\theta}{2}}|\le\frac{1}{|\sin\frac{\theta}{2}|}\]
同时$\dfrac{1}{n^2}$单调收敛于$0$。因此利用狄利克雷判别法我们知道该级数收敛。\\
(2)我们知道
\[\lim_{n\rightarrow+\infty}\sqrt[n]{\frac{n!}{n^n}}=\frac{1}{e}<1\]
因此根据比值判别法该级数收敛。\\
(3)同上题,此时有
\[\lim_{n\rightarrow+\infty}\sqrt[n]{\frac{n!a^n}{n^n}}=\frac{a}{e}\]
因此$a>e$时发散,$a<e$时收敛。当$a=e$时,我们有
\[\frac{a_{n+1}}{a_n}=\frac{e}{(1+\frac{1}{n})^n}>1\]
因此$a_n$并不趋于$0$,于是发散。\\
(4)我们有
\[\sqrt[n]{\frac{n^{\ln n}}{(\ln n)^n}}=\frac{n^{\frac{\ln n}{n}}}{\ln n}\]
取对数得
\[\frac{(\ln n)^2}{n}-\ln\ln n\]
显然当$n$趋于无穷时上式趋于负无穷,因此原式趋于$0$。因此该级数收敛。\\
(5)显然我们有
\[\lim_{n\rightarrow+\infty}\frac{n^{\frac{3}{2}}(\ln n)^p}{1+n^2}=\lim_{n\rightarrow+\infty}\frac{(\ln n)^p}{\frac{1}{n^\frac{3}{2}}+n^\frac{1}{2}}=0\]
且我们知道级数$\dfrac{1}{n^\frac{3}{2}}$收敛,因此该级数收敛。\\
(6)对任意正数$\epsilon$,我们有
\[\lim_{n\rightarrow+\infty}\frac{n^p(\ln n)^q}{n^p}=+\infty,\lim_{n\rightarrow+\infty}\frac{n^{p-\epsilon}(\ln n)^q}{n^p}=0\]
因此当$p\le1$时,该级数发散,当$p>1$时,该级数收敛。\\
(7)让$\dfrac{1}{n^\beta}$去除级数项,我们有
\[n^{\beta-\alpha}-n^\beta\sin\frac{1}{n^\alpha}=\frac{1}{3!}\frac{n^\beta}{n^{3\alpha}}-\frac{1}{5!}\frac{n^\beta}{n^{5\alpha}}+\cdots\]
当$\alpha>\dfrac{1}{3}$时,取$\beta=1+\epsilon,\epsilon>0$使得$\beta<3\alpha$,显然此时该级数收敛。当$\alpha=\dfrac{1}{3}$时,该极限等于$1$,与调和级数同敛散,因此发散。当$\alpha
<\dfrac{1}{3}$时,我们知道$x-\sin x$单调递增,且$\dfrac{1}{n^\alpha}>\dfrac{1}{n^\frac{1}{3}}$,因此该级数发散。\\
综上所述,$\alpha>\dfrac{1}{3}$级数收敛,$\alpha\le\dfrac{1}{3}$级数发散。\\
(8)显然当$p>1$时,该级数绝对收敛,因为此时有
\[\frac{1}{1^p}+\frac{1}{2^{2p}}+\cdots<1+\frac{1}{2^p}+\cdots\]
而右边收敛。当$p=1$是,我们知道
\[\frac{1}{(2n)^2}<\frac{1}{(2n-1)(2n+1)}=\frac{1}{2}(\frac{1}{2n-1}-\frac{1}{2n+1})\]
因此我们有
\begin{align*}
	&1-\frac{1}{2^2}+\frac{1}{3}-\cdots\\
	&>\frac{1}{2}+\frac{1}{3}+\frac{1}{5}+\cdots\\
	&>\frac{1}{2}+\frac{1}{4}+\frac{1}{6}+\cdots\\
	&=\frac{1}{2}(1+\frac{1}{2}+\frac{1}{3}+\cdots)\rightarrow+\infty
\end{align*}
因此该级数发散。鉴于我们有
\[(\frac{1}{(2n-1)^x}-\frac{1}{(2n)^{2x}})'=\frac{x}{(2n-1)^{x+1}}(\frac{2}{(2n^x)}(\frac{2n-1}{2n})^{x+1}-1)<0\]
\[\frac{1}{(2n+1)^p}>\frac{1}{(2n+1)},p<1\]
因此当$p<1$时该级数部分和大于$p=1$的情形,且后者发散,因而$p<1$时发散。\\
综上所述,$p>1$级数收敛,$p\le1$级数发散。\\
(9)此即(1)中$\theta$取$1$的情况,根据狄利克雷判别法该级数收敛。\\
2.\\
将$f(n)$写作$a_n$,我们有
\begin{align*}
	&a_1+a_2+a_3+\cdots+a_{2^n-1}\\
	&\le a_1+2a_2+4a_4+\cdots+2^{n-1}a_{2^{n-1}}	
\end{align*}
以及
\begin{align*}
	&a_1+a_2+a_3+\cdots+a_{2^n}\\
	&\ge a_2+2a_4+4a_8+\cdots+2^{n-1}a_{2^n}
\end{align*}
因此显然两个级数同敛散。\\
3.\\
我们有
\[\lim_{n\rightarrow+\infty}\frac{\ln(1+a_n)}{a_n}=1\]
因此显然两个级数同敛散。
\end{document}









