\documentclass[utf8]{ctexart}
\usepackage{graphicx}
\usepackage{amsmath}
\usepackage{amssymb}
\usepackage{siunitx}
\makeatletter
\newcommand{\rmnum}[1]{\romannumeral #1}
\newcommand{\Rmnum}[1]{\expandafter\@slowromancap\romannumeral #1@}
\makeatother
\title{统计力学作业5}
\author{郑子诺,物理41}
\date{\today}
\begin{document}
\maketitle
\noindent 补充题1:\\
玻色子;玻色子;费米子;玻色子;玻色子;\\
费米子;费米子;玻色子;费米子\\
补充题2:\\
\[\{\{4,0\},\{3,1\},\{2,2\},\{1,3\},\{0,4\}\},\{1,16,36,16,1\},\{2,2\}\]
\[\{\{4,0\},\{3,1\},\{2,2\},\{1,3\},\{0,4\}\},\{35,80,100,80,35\},\{2,2\}\]
6.5\&6.6\\
对于经典粒子,我们有
\[\Omega\{a_i,a'_i\}=\left(\frac{N!}{\prod a_i!}\prod\omega_i^{a_i}\right)\left(\frac{N'!}{\prod a'_i!}\prod\omega{a'_i}^{a'_i}\right)\]
求对数并利用斯特林公式近似得
\[\ln\Omega\{a_i,a'_i\}=N\ln N-N-\sum(a_i\ln a_i-a_i)+\sum a_i\ln\omega_i+N'\ln N'-N'-\sum(a'_i\ln a'_i-a'_i)+\sum a'_i\ln\omega'_i\]
利用拉格朗日乘子法,主要到$\alpha,\alpha'$不同,但是$\beta$相同,因为能量约束条件涉及两种粒子,分别对$a_l,a'_l$求导取$0$得到
\[a_l=\omega_le^{-\alpha-\beta \epsilon_l},a'_l=\omega'_le^{-\alpha'-\beta \epsilon'_l}\]
由于两种粒子近独立,类似的,对于费米子和玻色子有
\[a_l=\frac{\omega_l}{e^{\alpha+\beta \epsilon_l}\pm1},a'_l=\frac{\omega'_l}{e^{\alpha'+\beta \epsilon'_l}\pm1}\]
其中$+$为费米子,$-$为玻色子。\\
根据热力学第三定律,在温度趋于零时系统趋于基态,观察可知$\beta\rightarrow+\infty$可以满足这一条件,因此$\beta$和温度有关。\\
特别的,对于玻尔兹曼分布,$\alpha$是公有的,粒子分布只与能量有关,$\beta\rightarrow+\infty$时所有粒子将处于能量最低态;对于费米狄拉克分布,$\beta\rightarrow+\infty$时指数上为正的态上将没有粒子,反之粒子数等于简并数,由此可见粒子占据一个特定的能量下的每一个态;对于玻色爱因斯坦分布,选定基态作为能量零点,显然$\alpha\ge0$,否则基态粒子数为负,当$\beta\rightarrow+\infty$时除了基态外所有态上没有粒子,因此所有粒子处于基态。
\end{document}