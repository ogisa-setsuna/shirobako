\documentclass[utf8]{ctexart}
\usepackage{graphicx}
\usepackage{amsmath}
\usepackage{amsthm}
\usepackage{mathabx}
\usepackage{braket}
\title{费曼3作业2}
\author{郑子诺,物理41}
\date{\today}
\begin{document}
\maketitle
71.4\\
(a)Because the photons are Bosons,we have:
\[P_{emission}=|\braket{n+1|n}|^2=(n+1)P_0\]
\[P_{absorption}=|\braket{n-1|n}|^2=nP_0\]
The emission and absorption build an equilibrium:
\[N_2P_{emission}=N_1P_{absorption},N_1P_{emission}=N_0P_{absorption}\]
\[\therefore\frac{N_2}{N_1}=\frac{N_1}{N_0}=\frac{n(\omega)}{n(\omega)+1}\] 
(b)Use Boltzmann distribution we have:
\[\frac{N_2}{N_1}=\frac{N_1}{N_0}=e^{-\frac{\Delta E}{kT}}\]
\[\therefore n(\omega)=\frac{1}{e^{\frac{\Delta E}{kT}}-1}\]
(c)\\
\[\hbar\omega\gg kT,n(\omega)\approx e^{-\frac{\hbar\omega}{kT}}\]
\[\hbar\omega\ll kT,n(\omega)\approx \frac{kT}{\hbar\omega}\]
71.6\\
(a)Because the stimulated emission produces photons at exactly the same state as the one induces,the direction of the photons is obviously the same.\\
(b)Impossible.Because neutrinos have spin of one-half that they are fermions,they couldn't be the same state as a result of Pauli exclusion principle.\\
71.9\\
(a)The recoil breaks up the nucleus,which distinguishes the one proton interacted.Therefore the probability is:
\[P(\theta)=|f_1(\theta)+f_1(\theta)|^2+|f_2(\theta)+f_2(\theta)|^2\]
(b)The nucleus remains intact,which means the final states of two kinds of scattering are the same.Hence the probability is:
\[P_(\theta)=|f_1(\theta)+f_1(\theta)+f_2(\theta)+f_2(\theta)|^2\]
(c)The probability of the case in (b) have an interference term,which is probably positive and gives more scattering,otherwise it gives less scattering.\\
71.11\\
(a)Because the velocity of proton in center-of-mass system is the same as the velocity of center-of-mass,we have
\[\alpha=\frac{\theta}{2}\]
(b)Obviously it's
\[g(\theta)-f'(\pi-\theta)\]
(c)The scattering probability of up-up and down-down is
\[2|f(\theta)-f(\pi-\theta)|^2\]
The scattering probability of up-down and down-up is
\[|f'(\theta)-g(\pi-\theta)|^2+|g(\theta)-f'(\pi-\theta)|^2\]
Because the probability of each case is $\dfrac{1}{4}$,we have:
\[P(\theta)=|f(\theta)-f(\pi-\theta)|^2+\frac{1}{2}(|f'(\theta)-g(\pi-\theta)|^2+|g(\theta)-f'(\pi-\theta)|^2)\]
(d)Simplify the result of (c) we have
\[P(\theta)=|f(\theta)-f(\pi-\theta)|^2+\frac{1}{2}(|f(\theta)|^2+|f(\pi-\theta)|^2)=\frac{5}{4}|f(\theta)-f(\pi-\theta)|^2+\frac{1}{4}|f(\theta)+f(\pi-\theta)|^2\]
(e)\[A=\frac{5}{4},B=\frac{1}{4}\]
\end{document}

















