\documentclass[utf8]{ctexart}
\usepackage{graphicx}
\usepackage{amsmath}
\usepackage{amssymb}
\usepackage{braket}
\title{线代作业11}
\author{郑子诺,物理41}
\date{\today}
\begin{document}
\maketitle
\noindent1.\\
\[\braket{A,B}=Tr(A^TB)=\sum_{j=1}^n\sum_{i=1}^ma_{ij}b_{ij}=Tr(B^TA)=\braket{B,A}\]
\[\braket{A,cB+C}=Tr(A^T(cB+C))=cTr(A^TB)+Tr(A^TC)=c\braket{A,B}+\braket{A,C}\]
\[\braket{A,A}=Tr(A^TA)=\sum_{j=1}^n\sum_{i=1}^ma^2_{ij}>0,\text{ if }A\neq0\]
2.\\
\[\braket{x,y}=X^TGY\]
\[X=PX',Y=PY'\]
\[\therefore\braket{x,y}=X^TGY=X'^TP^TGPY'\rightarrow G'=P^TGP\]
3.\\
若$\alpha_1,\dots,\alpha_n$线性相关,那么$G$存在一列向量是其他列向量的线性组合,这是因为若
\[\alpha_k=c_1\alpha_1+\cdots+c_{k-1}\alpha_{k-1}+c_{k+1}\alpha_{k+1}+\cdots+c_n\alpha_n\]
那么
\[\braket{\alpha_i,\alpha_k}=c_1\braket{\alpha_i,\alpha_1}+\cdots+c_{k-1}\braket{\alpha_i,\alpha_{k-1}}+c_{k+1}\braket{\alpha_i,\alpha_{k+1}}+\cdots+c_n\braket{\alpha_i,\alpha_{n}}\]
即
\[g_k=c_1g_1+\cdots+c_{k-1}g_{k-1}+c_{k+1}g_{k+1}+\cdots+c_ng_n\]
显然此时$G$不可逆。若$\alpha_1,\dots,\alpha_n$线性无关而$G$不可逆,那么存在一列向量是其他列向量的线性组合,我们有
\[g_k=c_1g_1+\cdots+c_{k-1}g_{k-1}+c_{k+1}g_{k+1}+\cdots+c_ng_n\]
\[\braket{\alpha_i,\alpha_k}=c_1\braket{\alpha_i,\alpha_1}+\cdots+c_{k-1}\braket{\alpha_i,\alpha_{k-1}}+c_{k+1}\braket{\alpha_i,\alpha_{k+1}}+\cdots+c_n\braket{\alpha_i,\alpha_{n}}\]
于是
\[\braket{\alpha_i,\alpha_k}=\braket{\alpha_i,c_1\alpha_1+\cdots+c_{k-1}\alpha_{k-1}+c_{k+1}\alpha_{k+1}+\cdots+c_n\alpha_{n}}\]
由于此时$\alpha_1,\dots,\alpha_n$线性无关,形成一组基,那么上式表明$\alpha_k$与$c_1\alpha_1+\cdots+c_{k-1}\alpha_{k-1}+c_{k+1}\alpha_{k+1}+\cdots+c_n\alpha_{n}$对任何向量做的内积都相同,于是两者相同,与线性无关矛盾。因而$G$可逆,证毕。\\
4.\\
设$A$为
\[\begin{bmatrix}
	a&b\\
	b&c
\end{bmatrix}\]
我们有
\[X^TAX=ax^2+2bxy+cy^2\]
固定$y\neq0$,我们有$\Delta=4y^2(b^2-ac)<0$,要求了$\det A>0$。同理固定$x\neq0$也要求$\det A>0$。由于$X^TAX>0,X\neq0$,因此一定有$a,c>0$,于是$TrA=a+c>0$。\\
若已有$TrA>0,\det A>0$,那么显然由于$\det A=ac-b^2>0\rightarrow ac>0$,我们有$a>0,c>0$,根据上述讨论这显然满足$X^TAX>0,X\neq0$。证毕。\\
5.\\
\[\Vert x+y\Vert^2=\Vert x\Vert^2+\Vert y\Vert^2+2\braket{x,y}\]
因此满足勾股定理当且仅当$\braket{x,y}=0$,即$x,y$正交。\\
6.\\
\begin{align*}
	\braket{x,y}&=\sum_{i=1}^n\sum_{j=1}^nx_iy_j\braket{\alpha_i,\alpha_j}\\
	&=\sum_{i=1}^n\sum_{j=1}^nx_iy_j\delta_{ij}\\
	&=\sum_{i=1}^nx_iy_i
\end{align*}
又由于
\[x_i=\sum_{j=1}^nx_j\braket{\alpha_i,\alpha_j}=\braket{\alpha_i,x_i},y_i=\braket{\alpha_i,y_i}\]
所以
\[\braket{x,y}=\sum_{i=1}^n\braket{x,\alpha_i}\braket{\alpha_i,y}\]
7.\\
利用Gram-Schmidt正交化,我们有
\[\alpha_1=1\]
\[\alpha_2=x-\frac{1}{2}\]
\[\alpha_3=x^2-x+\frac{1}{6}\]
\[\alpha_4=x^3-\frac{3}{2}x+\frac{3}{5}x-\frac{1}{20}\]
归一化得
\[e_1=1,e_2=2\sqrt{3}(x-\frac{1}{2}),e_3=6\sqrt{5}(x^2-x+\frac{1}{6}),e_4=20\sqrt{7}(x^3-\frac{3}{2}x+\frac{3}{5}x-\frac{1}{20})\]
8.\\
令
\[A=\begin{bmatrix}
	0&1&1\\
	0&1&-1\\
	0&0&1
\end{bmatrix}\]
显然满足题设条件。\\
9.\\
显然
\[\begin{bmatrix}
	-2\\
	1\\
	0
\end{bmatrix},\begin{bmatrix}
-2\\
0\\
1
\end{bmatrix}\]
是$U$的一组基。将其正交归一为
\[\frac{1}{\sqrt{5}}\begin{bmatrix}
	-2\\
	1\\
	0
\end{bmatrix},\frac{1}{3\sqrt{5}}\begin{bmatrix}
-2\\
-4\\
5
\end{bmatrix}\]
因此$p'$为
\[p'=\sum_{i=1}^2\braket{\alpha_i,p}\alpha_i=\frac{1}{9}\begin{bmatrix}
	-2\\
	-4\\
	5
\end{bmatrix}\]
10.\\
(a)基矢为$\dfrac{1}{\sqrt{n}}I_n$。因此
\[cTr(I_n^TA)=0\rightarrow TrA=0\Rightarrow U^\bot=\{A|TrA=0\}\]
\[p_{U}A=\frac{1}{n}(Tr(I_n^TA))I_n=\frac{TrA}{n}I_n\]
(b)显然正交归一基矢为
\[(e_i)_{jk}=\delta_{ij}\delta_{jk}\]
因此有
\[\sum_{i=1}^nc_iTr(e_i^TA)e_i=0\rightarrow a_{ii}=0\Rightarrow U^\bot=\{A|a_{ii}=0\}\]
\[p_{U}A=\sum_{i=1}^nTr(e_i^TA)e_i=\sum_{i=1}^na_{ii}e_i\]
(c)显然正交归一基矢为仅在上三角区域有一个分量为$1$其余为$0$的矩阵。显然有
\[U^\bot=\{A|a_{ij}=0,j\ge i\}\]
\[(p_{U}A)_{ij}=\begin{cases}
	0&j<i\\
	a_{ij}&j\ge i
\end{cases}\]
11.\\
令$p_U$为$U$的正交投影。令$\alpha\in (U^\bot)^\bot$,那么
\[\braket{\alpha-p_U\alpha,\beta}=0,\forall\beta\in U^\bot\]
又由于$\alpha-p_U\alpha\in U^\bot$,那么
\[\alpha-p_U\alpha=0\rightarrow\alpha=p_U\alpha\in U\Rightarrow (U^\bot)^\bot\subseteq U\]
又有
\[\braket{\alpha,\beta}=0,\forall\alpha\in U,\beta\in U^\bot\Rightarrow U\subseteq (U^\bot)^\bot\]
因此
\[U=(U^\bot)^\bot\]
12.\\
显然我们有
\[\braket{\alpha_1+\alpha_2,\beta}=\braket{\alpha_1,\beta}+\braket{\alpha_2,\beta}=0,\forall\alpha_i\in U_i,\beta\in U_1^\bot\cap U_2^\bot\]
因此
\[U_1^\bot\cap U_2^\bot\subseteq(U_1+U_2)^\bot\]
又有
\[\braket{\alpha_1+\alpha_2,\beta}=0,\forall\alpha_i\in U_i,\beta\in (U_1+U_2)^\bot\]
分别令$\alpha_1=0,\alpha_2=0$我们得到
\[\braket{\alpha_i,\beta}=0,\forall\alpha_i\in U_i,\beta\in (U_1+U_2)^\bot\]
这意味着
\[\beta\in U_1^\bot\cap U_2^\bot\Rightarrow (U_1+U_2)^\bot\subseteq U_1^\bot\cap U_2^\bot\]
于是
\[(U_1+U_2)^\bot=U_1^\bot\cap U_2^\bot\]
\end{document}




















