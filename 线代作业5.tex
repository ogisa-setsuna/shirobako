\documentclass{ctexart}
\usepackage{graphicx}
\usepackage{amsmath}
\usepackage{amsthm}
\usepackage{amssymb}
\usepackage{braket}
\graphicspath{{C:/Users/18967/Desktop/latex/figure/}}
\title{线代作业5}
\author{郑子诺,物理41}
\date{\today}
\begin{document}
\maketitle
\noindent1.\\
(1):
\[
\begin{vmatrix}
	1&2&3\\
	0&1&4\\
	5&6&0
\end{vmatrix}
=
0+40+0-24-0-15=1\]
(2):
\[\begin{vmatrix}
	1&2&3&4\\
	0&1&0&2\\
	1&0&1&1\\
	2&3&0&1
\end{vmatrix}
=1\begin{vmatrix}
	1&0&2\\
	0&1&1\\
	3&0&1
\end{vmatrix}
+1\begin{vmatrix}
	2&3&4\\
	1&0&2\\
	3&0&1
\end{vmatrix}
-2\begin{vmatrix}
	2&3&4\\
	1&0&2\\
	0&1&1
\end{vmatrix}
=-5+15-6=16\]
(3):
\[
\begin{vmatrix}
	1&2&3&4\\
	2&3&4&1\\
	3&4&1&2\\
	4&1&2&3
\end{vmatrix}
=40\begin{vmatrix}
	1&1&0&3\\
	2&1&0&-1\\
	3&0&-1&0\\
	1&0&0&0
\end{vmatrix}
=-40\begin{vmatrix}
	1&0&3\\
	1&0&-1\\
	0&-1&0
\end{vmatrix}
=160\]
2.\\
(1):\\
\[\begin{vmatrix}
a&b&\cdots&0\\
0&a&\cdots&0\\
\vdots&\vdots&\ddots&\vdots\\
b&0&\cdots&a
\end{vmatrix}
=a\begin{vmatrix}
	a&b&\cdots&0\\
	0&a&\cdots&0\\
	\vdots&\vdots&\ddots&\vdots\\
	0&0&\cdots&a
\end{vmatrix}+(-1)^{1+n}b\begin{vmatrix}
b&0&\cdots&0\\
a&b&\cdots&0\\
\vdots&\vdots&\ddots&\vdots\\
0&0&\cdots&b
\end{vmatrix}
=a^n+(-1)^{1+n}b^n\]
(2):\\
用归纳法。做归纳假设:
\[\det A_n=a_n+a_{n-1}x+\dots+a_1x^{n-1}+x^n\]
对于$n=1$显然成立。设对于$n=k-1$成立。对于$n=k$我们有
\[\det A_k=a_k(-1)^{n+1}\begin{vmatrix}
-1&0&\cdots&0\\
x&-1&\cdots&0\\
\vdots&\vdots&\ddots&\vdots\\
0&0&\cdots&-1
\end{vmatrix}+x\det A_{k-1}
=a_k+a_{k-1}x+\dots+a_1x^{k-1}+x^{k}\]
由数学归纳法知假设成立。
\[\therefore\begin{vmatrix}
	x&-1&\cdots&0\\
	0&x&\cdots&0\\
	\vdots&\vdots&\ddots&\vdots\\
	0&0&\cdots&-1\\
	a_n&a_{n-1}&\cdots&x+a_1
\end{vmatrix}=a_n+a_{n-1}x+\dots+a_1x^{n-1}+x^n\]
3.\\
(a):\\
对第$k$行乘一个数$c$
\[P_{ij}=\begin{cases}
	\delta_{ij}&\text{if } i\neq k\\
	c\delta_{ij}&\text{if } i=k
\end{cases}\]
\[\det P=c\]
第$k$行和第$k'$行交换
\[P_{ij}=\begin{cases}
	\delta_{ij}&\text{if } i\neq k,k'\\
	\delta_{k'j}&\text{if } i=k\\
	\delta_{kj}&\text{if }i=k'
\end{cases}\]
\[\det P=-1\]
第$k$行乘$c$加到第$k'$行
\[P_{ij}=\begin{cases}
	\delta_{ij}&\text{if } i\neq k\\
	c\delta_{kj}+\delta_{ij}&\text{if } i=k'
\end{cases}\]
\[\det P=1\]
以上是显然的。\\
(b):\\
对第$k$列乘一个数$c$
\[Q_{ij}=\begin{cases}
	\delta_{ij}&\text{if } j\neq k\\
	c\delta_{ij}&\text{if } j=k
\end{cases}\]
\[\det Q=c\]
第$k$列和第$k'$列交换
\[Q_{ij}=\begin{cases}
	\delta_{ij}&\text{if } j\neq k,k'\\
	\delta_{ik'}&\text{if } j=k\\
	\delta_{ik}&\text{if }j=k'
\end{cases}\]
\[\det Q=-1\]
第$k$列乘$c$加到第$k'$列
\[Q_{ij}=\begin{cases}
	\delta_{ij}&\text{if } j\neq k\\
	c\delta_{ik}+\delta_{ik'}&\text{if } j=k'
\end{cases}\]
\[\det Q=1\]
以上是显然的。\\
4.\\
\[|A|=\begin{vmatrix}
	B&C\\
	0&D
\end{vmatrix}
=\sum\limits_\sigma sgn(\sigma)a_{1\sigma(1)}a_{2\sigma(2)}\dots a_{n\sigma(n)}\]
\[\because i>k,a_{ij}=\begin{cases}
	0&\text{if } j\le n-k\\
	d_{(i-k)j}&\text{if }j>n-k
\end{cases}\]
\[\therefore\sigma(k+1),\sigma(k+2),\dots,\sigma(n)\notin 1,2,\dots,k\rightarrow\sigma(1),\sigma(2),\dots,\sigma(k)\in1,2,\dots,k\]
\[\therefore\sum\limits_\sigma sgn(\sigma)a_{1\sigma(1)}a_{2\sigma(2)}\dots a_{n\sigma(n)}=\sum\limits_\sigma sgn(\sigma)b_{1\sigma'(1)}b_{2\sigma'(2)}\dots b_{k\sigma'(k)}d_{k+1\sigma''(k+1)}d_{k+2\sigma''(k+2)}\dots d_{n\sigma''(n)}\]
\[\because\sigma=\{\sigma',\sigma''\},sgn(\sigma)=sgn(\sigma')sgn(\sigma'')\]
\[\therefore|A|=\begin{vmatrix}
	B&C\\
	0&D
\end{vmatrix}
=\sum\limits_{\sigma',\sigma''} sgn(\sigma')b_{1\sigma'(1)}b_{2\sigma'(2)}\dots b_{k\sigma'(k)}sgn(\sigma')d_{k+1\sigma''(k+1)}d_{k+2\sigma''(k+2)}\dots d_{n\sigma''(n)}
=|B||D|\]
\end{document}
















