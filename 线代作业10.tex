\documentclass[utf8]{ctexart}
\usepackage{graphicx}
\usepackage{amsmath}
\usepackage{amssymb}
\title{线代作业10}
\author{郑子诺,物理41}
\date{\today}
\begin{document}
\maketitle
\noindent1.
(a)显然,对于$J_m(0)$,每乘自己一次,整体就往右平移一列,因而有
\[J_m(0)^m=0\]
因此$N$足够大时,我们有
\[J_m(\lambda)^N=\sum_{k=0}^NJ_m(0)^k\lambda^{N-k}=\sum_{k=0}^{m-1}J_m(0)^k\lambda^{N-k}\]
或者显式地写出来
\[\begin{bmatrix}
	\lambda^N&\lambda^{N-1}&\cdots&\lambda^{N-m+2}&\lambda^{N-m+1}\\
	0&\lambda^N&\cdots&\lambda^{N-m+3}&\lambda^{N-m+2}\\
	\vdots&\vdots&\ddots&\vdots&\vdots\\
	0&0&\cdots&\lambda^N&\lambda^{N-1}\\
	0&0&\cdots&0&\lambda^N
\end{bmatrix}\]
由于$|\lambda|<1$,显然每一项在$N\rightarrow\infty$时都趋于零。因此有
\[\lim_{N\rightarrow\infty}J_m(\lambda)^N=0\]
(b)每个方阵都相似于约当标准型,因此有
\[A^N=P\begin{bmatrix}
	J_{m_{11}}(\lambda_1)^N&0&\cdots&0\\
	0&J_{m_{12}}(\lambda_1)^N&\cdots&0\\
	\vdots&\vdots&\ddots&\vdots\\
	0&0&\cdots&J_{m_{kn_k}}(\lambda_k)^N
\end{bmatrix}P^{-1}\]
显然每一项趋于$0$。\\
2.\\
(a)设存在,为$\alpha$,则有
\[\sum_{j=1}^n\hat{m_{ij}}a_j=-a_i\]
因而
\[\sum_{i=1}^n\sum_{j=1}^n\hat{m_{ij}}a_j=\sum_{j=1}^n\sum_{i=1}^n\hat{m_{ij}}a_j=\sum_{j=1}^na_j=-\sum_{i=1}^na_i\]
因此
\[\sum_{i=1}^na_i=0\]
同时若存在$a_i$不同号,我们有
\[|a_i|=|\sum_{j=1}^n\hat{m_{ij}}a_j|<\sum_{j=1}^n\hat{m_{ij}}|a_j|\]
因而
\[\sum_{j=1}^n|a_j|>\sum_{i=1}^n|a_i|\]
这显然不可能,因此$a_i$同号,由于其和为$0$,因此必为零向量,于是不存在该特征值。\\
(b)如果根子空间与特征子空间不一致,显然我们可以找到一个向量使得$(\widehat{M}-I)^m\alpha=0,(\widehat{M}-I)^{m-1}\alpha\neq0$。于是我们总可以写成
\[(\widehat{M}-I)^2\alpha=0,(\widehat{M}-I)\alpha\neq0,\alpha\neq0\]
由于前一题所证性质,$\beta=(\widehat{M}-I)\alpha$各分量同号。又有
\[\sum_{i=1}^nb_i=\sum_{i=1}^n(\sum_{j=1}^n\hat{m_{ij}}a_j-a_i)=0\]
因此$\beta=0$,矛盾。于是代数重数也为$1$。
\end{document}












