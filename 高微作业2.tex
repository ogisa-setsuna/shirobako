\documentclass[utf8]{ctexart}
\usepackage{graphicx}
\usepackage{amsmath}
\usepackage{mathabx}
\title{高微作业2}
\author{郑子诺,物理41}
\date{\today}
\begin{document}
\maketitle
\noindent1.\\
(1):\\
\[\because\frac{1}{2}=\frac{1+2+\dots+n}{n^2+n}<\frac{1}{n^2+1}+\frac{2}{n^2+2}+\dots+\frac{n}{n^2+n}<\frac{1+2+\dots+n}{n^2+1}=\frac{1}{2}\frac{n^2+n}{n^2+1}\]
\[\frac{1}{2}\le\lim_{n\rightarrow\infty}\frac{1}{n^2+1}+\frac{2}{n^2+2}+\dots+\frac{n}{n^2+n}\le\lim_{n\rightarrow\infty}\frac{1}{2}\frac{n^2+n}{n^2+1}=\frac{1}{2}\]
\[\therefore\lim_{n\rightarrow\infty}\frac{1}{n^2+1}+\frac{2}{n^2+2}+\dots+\frac{n}{n^2+n}=\frac{1}{2}\]
(2):\\
\[\lim_{n\rightarrow\infty}\sqrt{n^2+an+b}-n=\lim_{n\rightarrow\infty}\frac{an+b}{\sqrt{n^2+an+b}+n}=\lim_{n\rightarrow\infty}\frac{a+\frac{b}{n}}{\sqrt{1+\frac{a}{n}+\frac{b}{n^2}}+1}=\frac{a}{2}\]
2.\\
\[\lim_{n\rightarrow\infty}\sqrt[n]{n^k+a_{k-1}n^{k-1}+\dots+a_0}=\lim_{n\rightarrow\infty}(\sqrt[n]{n})^k\lim_{n\rightarrow\infty}\sqrt[n]{1+a_{k-1}\frac{1}{n}+\dots+\frac{a_0}{n^k}}=1\]
3.\\
(1):\\
显然,存在$N$,使得当$n>N$时有:
\[\frac{|a_{n+1}|}{|a_n|}<r<1\]
\[|a_n|=r^{n-n_0}|a_{n_0}|\rightarrow\lim_{n\rightarrow\infty}|a_n|=0\]
\[\therefore\lim_{n\rightarrow\infty}a_n=0\]
(2):\\
\[\frac{a^{n+1}}{(n+1)!}\frac{n!}{a^n}=\frac{a}{n}\]
\[\because\lim_{n\rightarrow\infty}\frac{a}{n}=0<1\]
\[\therefore\lim_{n\rightarrow\infty}\frac{a^n}{n!}=0\]
(3):\\
\[\frac{n^k}{a^{n+1}}\frac{a^n}{n^k}=\frac{1}{a}<1\]
\[\therefore\lim_{n\rightarrow\infty}\frac{n^k}{a^n}=0\]
(4):\\
\[\frac{(n+1)!}{(\frac{n+1}{q})^(n+1)}\frac{(\frac{n}{q})^n}{n!}=\frac{q}{(1+\frac{1}{n})^n}\]
\[\because\lim_{n\rightarrow\infty}(1+\frac{1}{n})^n=e,0<q<e\]
\[\therefore\lim_{n\rightarrow\infty}\frac{q}{(1+\frac{1}{n})^n}=\frac{q}{e}<1\]
\[\therefore\lim_{n\rightarrow\infty}\frac{n!}{(\frac{n}{q})^n}=0\]
4.\\
(1):\\
由均值不等式得:
\[x_{n}=\frac{1}{2}(x_{n-1}+\frac{k}{x_{n-1}})\ge\frac{1}{2}2\sqrt{k}=\sqrt{k},n\ge1\]
(2):\\
\[\because x_n\ge\sqrt{k}\]
\[\therefore\frac{1}{2}\frac{k}{x_n}\le\frac{x_n}{2}\]
\[\therefore x_{n+1}\le x_n,n\ge1\]
(3):\\
\[\because x_n\ge\sqrt{k},x_n\ge x_{n+1}\]
由单调有界序列收敛定理知$\{x_n\}_{n=1}^{\infty}$收敛。
(4):\\
\[x_n-x_{n+1}=\frac{(x_n-\sqrt{k})(x_n+\sqrt{k})}{2x_n}\ge\frac{\sqrt{k}}{x_n}(x_n-\sqrt{k})\]
\[\therefore\frac{x_{n+1}-\sqrt{k}}{x_n-\sqrt{k}}\le1-\frac{\sqrt{k}}{x_n}<1\]
\[\because x_n\ge x_{n+1}\]
\[\frac{x_{n+1}-\sqrt{k}}{x_n-\sqrt{k}}<q<1\]
\[\therefore\lim_{n\rightarrow\infty}x_n-\sqrt{k}=0\rightarrow\lim_{n\rightarrow\infty}x_n=\sqrt{k}\]
5.\\
(1):\\
由均值不等式得:
\[y_n=\frac{1}{2}(x_{n-1}+y_{n-1})\ge\sqrt{x_{n-1}y_{n-1}}=x_n,n\ge1\]
(2):\\
\[\because y_n\ge x_n\]
\[\therefore x_{n+1}=\sqrt{x_ny_n}\ge x_n,y_{n+1}=\frac{1}{2}(x_n+y_n)\le y_n\]
(3):\\
\[x_{n+1}\le y_n\le y_0=b,y_{n+1}\ge x_n\ge x_0=a\]
(4):\\
\[\because x_n\le b,x_n\le x_{n+1},y_n\ge a,y_n\ge y_{n+1}\]
由单调有界序列收敛定理知$\{x_n\}_{n=1}^{\infty},\{y_n\}_{n=1}^{\infty}$收敛。
(5):\\
\[y_{n+1}-x_{n+1}=\frac{1}{2}(\sqrt{y_n}-\sqrt{x_n})^2<\frac{y_n-x_n}{2}\]
\[\therefore\frac{y_{n+1}-x_{n+1}}{y_n-x_n}<\frac{1}{2}<1\]
\[\therefore\lim_{n\rightarrow\infty}y_n-x_n=0\rightarrow\lim_{n\rightarrow\infty}y_n=\lim_{n\rightarrow\infty}x_n\]
6.\\
(1):\\
设存在$N$使得$a_N>A$,则对于$n>N$有$a_n\ge a_N>A$,令$\epsilon=a_N-A$
\[\therefore\text{不存在}N_0\text{,因为当}n>max\{N_0,N\},|a_n-A|>\epsilon\]
与极限定义矛盾。
\[\therefore a_n\le A\]
(2):\\
下证$(1+\dfrac{1}{n})^n$递增:
\[\binom{k}{n+1}\frac{1}{(n+1)^k}=\frac{(1-\frac{1}{n+1})(1-\frac{2}{n+1})\cdots(1-\frac{k}{n+1})}{k!}>\frac{(1-\frac{1}{n})(1-\frac{2}{n})\cdots(1-\frac{k}{n})}{k!}=\binom{k}{n}\frac{1}{n^k}\]
\[\therefore(1+\frac{1}{n+1})^{n+1}>(1+\frac{1}{n})^n\]
下证$(1+\dfrac{1}{n})^{n+1}$递减:
\[\binom{k}{n+1}\frac{1}{(n)^k}>\binom{k}{n+1}\frac{1}{(n+1)^k}\frac{n+2}{n+1}\]
\[\rightarrow1>(\frac{n}{n+1})^{k-1}\frac{n(n+2)}{n+1}\]
因此成立。将(1)中结论用于递减序列有类似结果,因此:
\[\because\lim_{n\rightarrow\infty}(1+\frac{1}{n})^n=\lim_{n\rightarrow\infty}(1+\frac{1}{n})^{n+1}=e\]
\[\therefore(1+\frac{1}{n})^n\le e\le(1+\frac{1}{n})^{n+1}\]
(3):\\
显然,题中结论可由(2)中结论连乘得到。
\[\frac{(1+n)^n}{n^n}\frac{n^{n-1}}{(n-1)^{n-1}}\cdots\frac{1+1}{1}=\frac{(1+n)^n}{n!}\le e^n\le\frac{(1+n)^{1+n}}{n!}\]
\[\therefore\frac{(1+n)^n}{e^n}\le n!\le\frac{(1+n)^{1+n}}{e^n}\]
(4):\\
\[\frac{1+\frac{1}{n}}{e}\le\frac{\sqrt[n]{n!}}{n}\le\frac{1+\frac{1}{n}}{e}\sqrt[n]{n+1}\]
\[\because\lim_{n\rightarrow\infty}\sqrt[n]{n+1}=1\]
\[\therefore\lim_{n\rightarrow\infty}\frac{1+\frac{1}{n}}{e}=\frac{1}{e}\le\lim_{n\rightarrow\infty}\frac{\sqrt[n]{n!}}{n}\le\frac{1}{e}=\lim_{n\rightarrow\infty}\frac{1+\frac{1}{n}}{e}\sqrt[n]{n+1}\]
\[\therefore\lim_{n\rightarrow\infty}\sqrt[n]{\frac{n!}{n^n}}=\frac{1}{e}\]
\end{document}





















