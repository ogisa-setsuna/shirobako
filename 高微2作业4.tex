\documentclass[utf8]{ctexart}
\usepackage{graphicx}
\usepackage{amsmath}
\usepackage{amssymb}
\makeatletter
\newcommand{\rmnum}[1]{\romannumeral #1}
\newcommand{\Rmnum}[1]{\expandafter\@slowromancap\romannumeral #1@}
\makeatother
\makeatletter
\newsavebox\myboxA
\newsavebox\myboxB
\newlength\mylenA
\newcommand*\xoverline[2][0.75]{%
	\sbox{\myboxA}{$\m@th#2$}%
	\setbox\myboxB\null% Phantom box
	\ht\myboxB=\ht\myboxA%
	\dp\myboxB=\dp\myboxA%
	\wd\myboxB=#1\wd\myboxA% Scale phantom
	\sbox\myboxB{$\m@th\overline{\copy\myboxB}$}% Overlined phantom
	\setlength\mylenA{\the\wd\myboxA}% calc width diff
	\addtolength\mylenA{-\the\wd\myboxB}%
	\ifdim\wd\myboxB<\wd\myboxA%
	\rlap{\hskip 0.5\mylenA\usebox\myboxB}{\usebox\myboxA}%
	\else
	\hskip -0.5\mylenA\rlap{\usebox\myboxA}{\hskip 0.5\mylenA\usebox\myboxB}%
	\fi}
\makeatother
\newcommand{\bm}[1]{\boldsymbol{#1}}
\title{高微2作业4}
\author{郑子诺,物理41}
\date{\today}
\begin{document}
\maketitle
\noindent1.\\
根据条件
\[\lim_{|\bm{x}|\rightarrow+\infty}f(\bm{x})=+\infty\]
我们有$\exists M>0$使得
\[f(\bm{x})>f(\bm{0}),|\bm{x}|>M\]
取闭球$\xoverline{B_M(\bm{0})}$,对其使用最值定理知,存在点$(x_0,y_0)$使得
\[f(x,y)\ge f(x_0,y_0),(x,y)\in B_M(\bm{0})\]
对于$|x|>M$,我们有
\[f(\bm{x})>f(\bm{0})\ge f(x_0,y_0)\]
因此
\[f(x,y)\ge f(x_0,y_0),\forall(x,y)\in \mathbb{R}^2\]
2.\\
(1)由于$x^2+y^2$为连续映射,因而闭集的原像是闭集,而单点$1$是闭集,于是$S=\{(x,y)|x^2+y^2=1\}$是闭集。\\
(2)由于$S$是$\mathbb{R}^2$中的有界闭集,因而紧致,所以最值定理成立,于是存在$(x_0,y_0),(x_1,y_1)\in S$使得
\[f(x_0,y_0)\le f(x,y)\le f(x_1,y_1),\forall(x,y)\in S\]
3.\\
(1)正确。若可微,鉴于线性映射连续,我们有
\begin{align*}
	\lim_{\bm{h}\rightarrow\bm{0}}f(\bm{x+h})-f(\bm{x})&=\lim_{\bm{h}\rightarrow\bm{0}}(L(\bm{h})+\alpha(\bm{h}))\\[8pt]
	&=0+\lim_{\bm{h}\rightarrow\bm{0}}|\bm{h}|\frac{\alpha(\bm{h})}{|\bm{h}|}\\[8pt]
	&=0
\end{align*}
因而连续。\\
(2)正确。若可微,我们有
\begin{align*}
	\nabla_{\bm{v}}f&=\lim_{t\rightarrow0}\frac{f(\bm{x}+\bm{v}t)-f(\bm{x})}{t}\\[8pt]
	&=\lim_{t\rightarrow0}\frac{tL(\bm{v})+\alpha(\bm{v}t)}{t}\\[8pt]
	&=L(\bm{v})+\lim_{t\rightarrow0}|\bm{v}|\frac{\alpha(\bm{v}t)}{|\bm{v}|t}\\[8pt]
	&=L(\bm{v})
\end{align*}
(3)(4)错误。取一函数$f$为
\[f(x,y)=\begin{cases}
	\dfrac{|y|\sqrt{x^2+y^2}}{x} & x\neq0\\[8pt]
	0 &x=0
\end{cases}\]
其各个方向导数都存在,因为令$\bm{v}=(v\cos\theta,v\sin\theta)$,我们有
\[\nabla_{\bm{v}}f=\begin{cases}
	\dfrac{v|\sin\theta|}{\cos\theta} & \cos\theta\neq0\\[8pt]
	0 & \cos\theta=0
\end{cases}\]
但是该函数显然不连续,因为对于$x\neq0$,我们有
\[|f|\ge\frac{y^2}{|x|}\]
因而在$x$轴附近无界。同理,线性关系也不成立,因为
\[\partial_xf=\frac{|\sin\theta|}{\cos\theta}=0\]
\[\partial_yf=0\]
而$\nabla_{\bm{v}}f$并非皆为$0$。因此不成立。\\
4.\\
根据定义,令$\bm{v}=(v\cos\theta,v\sin\theta)$,我们有
\[\nabla_{\bm{v}}\sqrt{|x^2-y^2|}=\lim_{t\rightarrow0}\frac{\sqrt{v^2t^2|\cos2\theta|}}{t}=\frac{|t|v\sqrt{|\cos2\theta|}}{t}\]
鉴于$\dfrac{|t|}{t}$无极限,因此只有当$\cos2\theta=0$时才有方向导数,于是满足条件的所有方向为
\[\theta=\frac{\pi}{4},\frac{3\pi}{4},\frac{5\pi}{4},\frac{7\pi}{4}\]
5.\\
(1)当$x\neq0$时,我们有
\[\partial_xx^y=yx^{y-1},\partial_yx^y=x^y\ln x\]
(2)当$x\neq0$时,我们有
\[\partial_x\arctan\frac{y}{x}=-\frac{y}{x^2+y^2},\partial_y\arctan\frac{y}{x}=\frac{x}{x^2+y^2}\]
(3)当$x_i$不全为$0$时,我们有
\[\partial_i\sqrt{x_1^2+\cdots+x_n^2}=\frac{x_i}{\sqrt{x_1^2+\cdots+x_n^2}}\]
6.\\
(1)令$\bm{v}=(v\cos\theta,v\sin\theta)$,我们有
\[\nabla_{\bm{v}}f=\lim_{t\rightarrow0}\frac{v|t|\sqrt{|\sin\theta\cos\theta|}}{t}\]
由于$\dfrac{|t|}{t}$无极限,并不是所有方向都存在方向导数,于是不可微。\\
(2)微分为零映射,因为首先$f(\bm{0})$显然为$0$,我们又有
\[\frac{|f(x,y)|}{\sqrt{x^2+y^2}}\le\sqrt{x^2+y^2}\]
后者在$|\bm{x}|\rightarrow0$的极限下趋于$0$,因此根据夹逼定理前项极限为$0$,因此我们有
\[f(x,y)=f(0,0)+0(\bm{h})+\alpha(\bm{h})\]
其中
\[\lim_{\bm{h}\rightarrow\bm{0}}\frac{\alpha(\bm{h})}{|\bm{h}|}=\lim_{|\bm{x}|\rightarrow0}\frac{f(x,y)}{\sqrt{x^2+y^2}}=0\]
因此可微,且根据微分的唯一性,微分正是零映射。\\
(3)不一定。取$g$为$\sqrt{2|xy|}$,显然有
\[\sqrt{x^2+y^2}\ge\sqrt{2|xy|}=g\]
而由(1)知$g$在$(0,0)$处不可微,因此不一定可微。
\end{document}

























