\documentclass[utf8]{ctexart}
\usepackage{graphicx}
\usepackage{amsmath}
\title{高等微积分第一次作业}
\author{郑子诺,物理41}
\date{\today}
\begin{document}
\maketitle
\noindent 1.\\
(1):\\
先证明充分性:\\
由条件映射$f:X\to Y$是双射可知,$\forall y\in Y$,存在唯一的$x\in X$与之对应,因此可以构造映射$f^{-1}:Y\to X$,使得每一个$y$与相应的$x$对应。显然,此时
$$f^{-1}\circ f(x)=f^{-1}(f(x))=f^{-1}(y)=x\rightarrow f^{-1}\circ f=id_{X}$$
同理可知$$f\circ f^{-1}(y)=id_{Y}$$
充分性证毕。\\
再证明必要性:\\
假设存在映射$f^{-1}:Y\to X$使得
$$f^{-1}\circ f=id_{X},f\circ f^{-1}(y)=id_{Y}$$
若存在不同的$x_1,x_2$使得$f(x_1)=f(x_2)$,则
$$x_1=f^{-1}\circ f(x_1)=f^{-1}(f(x_1))=f^{-1}(f(x_2))=f^{-1}\circ f(x_2)=x_2$$
矛盾,因此$f(x)$是单射。\\
若对于$y\in Y$没有$x$与之对应,则
$$y=f\circ f^{-1}(y)=f(f^{-1}(y))=f(x')$$
矛盾,因此$f(x)$是满射。\\
综上,$f(x)$是双射。必要性证毕。\\
Q.E.D.\\
(2):\\
由于$f:X\to Y,g:Y\to Z$是双射,因此$\forall z\in Z$,存在唯一的$\forall y\in Y$与之对应,对于这个$y$,存在唯一的$\forall x\in X$与之对应,由此可知复合函数\\$g\circ f:X\to Z$是双射。\\
又由复合的结合律可知
$$(f^{-1}\circ g^{-1})\circ (g\circ f)=f^{-1}\circ (g^{-1}\circ g)\circ f=f^{-1}\circ id_{Y}\circ f=f^{-1}\circ f=id_{X}$$
同理可得$$(g\circ f)\circ (f^{-1}\circ g^{-1})=id_{Z}$$
于是$$(g\circ f)^{-1}=f^{-1}\circ g^{-1}$$
Q.E.D.\\
2.\\
(1):\\
设$\alpha=inf A,\beta=inf B$,不失一般性,令$min\{\alpha,\beta\}=\alpha$,则有
$$\forall a\in A,a\ge \alpha$$
$$\forall b\in B,b\ge \beta\ge \alpha$$
$$\text{如果}\gamma>\alpha,\text{则}\gamma \text{不是}A\text{的下界,亦即不是}A\cup B\text{的下界。}$$
于是$inf(A\cup B)=min\{inf A,inf B\}$。\\
同理$sup(A\cup B)=min\{sup A,sup B\}$。\\
Q.E.D.\\
(2):\\
同(1)设$\alpha=inf A,\beta=inf B,max\{\alpha,\beta\}=\beta$,则有
$$\forall x\in A\cap B,x\in B,\text{即有}x\ge\beta$$
所以$inf(A\cap B)\ge max\{inf A,inf B\}$。\\
同理$sup(A\cap B)\le max\{sup A,sup B\}$。\\
Q.E.D.\\
3.\\
构造函数:$g(x)=\frac{f(x)+f(-x)}{2},h(x)=\frac{f(x)-f(-x)}{2}$\\
显然$$g(-x)=g(x),h(-x)=-h(x),f(x)=g(x)+h(x)$$
Q.E.D.\\
4.\\
设$x=0$,则$f(0)=2f(0),f(0)=0$。\\
设存在$x\neq 0$,使得$f(x)\neq 0$,则
$$f(2x)=2f(x),f(4x)=2f(2x),...$$
构成等比数列,其绝对值以$2$为倍数增加。\\
设$|f(x)|=a$,对于任一正数$M$,存在$n>\log_2\frac{M}{a}$使得$|f(2^nx)|>M$,与题设条件矛盾。\\
所以$\forall x,f(x)=0$。\\
5.\\
首先证明整数可以提出函数:
$$f(2x)=f(x)+f(x)=2f(x),f(nx)=f((n-1)x)+f(x),f(-nx)=f(-(n-1)x)-f(x)$$
由归纳法得$$f(px)=pf(x),p\text{是整数}$$
再证明正整数的倒数可以提出函数:
$$f(x)=2f(\frac{1}{2}x),f(x)=f(\frac{1}{n}x)+f(\frac{n-1}{n}x)=nf(\frac{1}{n}x)$$
可得$$f(\frac{1}{q}x)=\frac{1}{q}f(x),q\text{是正整数}$$
综上,对于任一有理数$\frac{p}{q}$,有
$$f(\frac{p}{q})=\frac{p}{q}f(1)$$
令$a=f(1)$,则对于有理数$f(x)=ax$。\\
Q.E.D.\\
6.\\
(1):\\
运用三角不等式得
$$\forall \epsilon>0,\text{存在}N\text{使得所有}n>N,||A|-|a_n||<|A-a_n|<\epsilon$$
所以$\lim_{n\to \infty}|a_n|=|A|$。\\
Q.E.D.\\
(2):\\
由题设条件知,$\text{存在}N\text{使得所有}n>N_0,|A-a_n|<A$,此时$a_n>0$。因此$\sqrt{a_n}$有意义。\\
于是有
$$\forall \epsilon>0,\text{存在}N>N_0\text{使得所有}n>N,|\sqrt{A}-\sqrt{a_n}|<\frac{|A-a_n|}{\sqrt{A}+\sqrt{a_n}}<\frac{\epsilon}{\sqrt{A}}$$
所以$\lim_{n\to \infty}\sqrt{a_n}=\sqrt{A}$\\
Q.E.D.
\end{document}

















