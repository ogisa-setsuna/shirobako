\documentclass[utf8]{ctexart}
\usepackage{graphicx}
\usepackage{amsmath}
\usepackage{amssymb}
\title{线代作业9}
\author{郑子诺,物理41}
\date{\today}
\begin{document}
\maketitle
\noindent1.\\
(a)特征多项式
\[f(\lambda)=(\lambda-3)(\lambda+1)^2\]
特征值
\[\lambda=-1,3\]
特征子空间分别为
\[c\begin{bmatrix}
	1\\
	2\\
	1
\end{bmatrix},c\begin{bmatrix}
1\\
2\\
2
\end{bmatrix}\]
根子空间分别为
\[a\begin{bmatrix}
1\\
1\\
0
\end{bmatrix}+b\begin{bmatrix}
-1\\
0\\
1
\end{bmatrix},c\begin{bmatrix}
1\\
2\\
2
\end{bmatrix}\]
显然不能对角化。\\
(b)
特征多项式
\[f(\lambda)=\lambda^2(\lambda-2)^2\]
特征值
\[\lambda=0,2\]
特征子空间分别为
\[c\begin{bmatrix}
	0\\
	1\\
	0\\
	1
\end{bmatrix},c\begin{bmatrix}
1\\
0\\
0\\
1
\end{bmatrix}\]
根子空间分别为
\[a\begin{bmatrix}
	1\\
	0\\
	0\\
	0
\end{bmatrix}+b\begin{bmatrix}
0\\
1\\
0\\
1
\end{bmatrix},a\begin{bmatrix}
1\\
0\\
1\\
0
\end{bmatrix}+b\begin{bmatrix}
1\\
0\\
0\\
1
\end{bmatrix}\]
显然不能对角化。\\
2.\\
(a)
若满足$V_1\cap V_2=\{0\}$,则有
\[\alpha=\beta_1+\beta_2,\alpha\in V_1+V_2,\beta_i\in V_i\]
若展开不唯一,则有
\[\alpha=\beta'_1+\beta'_2,\alpha\in V_1+V_2,\beta'_i\in V_i\]
\[(\beta'_1-\beta_1)=-(\beta'_2-\beta_2)\]
显然等式左右分别为$V_1,V_2$中的向量,若不为$0$,则与$V_1\cap V_2=\{0\}$矛盾。\\
若$V_1+V_2=V_1\oplus V_2$,则若$\alpha\in V_1\cap V_2$,有
\[\alpha=\alpha+0=0+\alpha\]
由表达式的唯一性可知$\alpha=0$。\\
(b)令
\[V_1=c\begin{bmatrix}
	1\\
	0
\end{bmatrix},V_2=c\begin{bmatrix}
0\\
1
\end{bmatrix},V_3=c\begin{bmatrix}
1\\
1
\end{bmatrix}\]
显然题设条件成立,但是此时有
\[\dim(V_1+V_2+V_3)=2<\dim V_1+\dim V_2+\dim V_3\]
因此不是直和。\\
(c)令
\[\alpha_1+\alpha_2+\alpha_3=0,\alpha_i\in V_i\]
根据题设条件我们知道,设$\alpha_j\neq0$,则有
\[\alpha_j=-\alpha_1+\cdots+\alpha_{j-1}\]
而等式两边分别属于$V_j$和$V_1+\cdots+V_{j+1}$,根据条件,必须为$0$。因此我们证明了线性无关性,即三者之和为直和。\\
3.\\
把$A$看做一个线性变换$T$在一组基下的表示矩阵。分别取$V_1,V_2$的一组基,由直和的性质可知合并起来是$V$的一组基。再根据不变子空间的性质很容易看出,此时$T$的表示矩阵一定长成这个样子L:
\[\begin{bmatrix}
	B&0\\
	0&C
\end{bmatrix}\]
我们知道变换一组基相当于相似变换,因而证毕。\\
4.\\
通过归纳法。显然$n=1$不用证。设$n=k-1$成立。下证$n=k$成立。因为是复方阵,所以特征多项式必有一根,取出该特征值的特征向量并将它扩成一组基。同样的,我们还是把复方阵当做一个线性变换$T$在一组基下的表示矩阵。显然,此时的表示矩阵为
\[\begin{bmatrix}
	\lambda&\alpha\\
	0&B
\end{bmatrix}\]
$B$为$k-1$阶复方阵,因此可以通过相似变换变成上三角矩阵,这相当于$PBP^{-1}=B'$。于是我们有
\[\begin{bmatrix}
	1&0\\
	0&P
\end{bmatrix}\begin{bmatrix}
\lambda&\alpha\\
0&B
\end{bmatrix}\begin{bmatrix}
1&0\\
0&P^{-1}
\end{bmatrix}=\begin{bmatrix}
\lambda&\alpha\\
0&PBP^{-1}
\end{bmatrix}=\begin{bmatrix}
\lambda&\alpha\\
0&B'
\end{bmatrix}\]
利用归纳法得证。\\
5.\\
(a)我们还是把复方阵当做一个线性变换$T$在一组基下的表示矩阵。先对空间进行根子空间分解。定义投影算符$E_i^2=E_i$,并使得$E_i\alpha=\alpha_i,\alpha_i\in W_i$。显然此时有
\[E_1+\cdots+E_k=I,E_iE_j=0,i\neq j\]
且有
\[TE_i\alpha=T\alpha_i=E_iT\alpha_i=E_iT\alpha\]
因此
\[TE_i=E_iT\]
且投影算符是线性变换。令$D=\lambda_1E_1+\cdots+\lambda_kE_k$,显然$D$可对角化,因为此时$W_i$就是$\lambda_i$的特征子空间,这是显然的。令$N=T-D$我们有
\[N=(T-\lambda_1I)E_1+\cdots+(T-\lambda_kI)E_k\]
以及
\[N^r=(T-\lambda_1I)^rE_1+\cdots+(T-\lambda_kI)^rE_k\]
这可有$E_i$性质轻松得到。只要$r$足够大,根据根子空间定义,必有$N^r=0$。同时我们有
\[DN=\lambda_1(T-\lambda_1I)E_1+\cdots+\lambda_k(T-\lambda_kI)E_k=ND\]
证毕。\\
(b)
\[D=\begin{bmatrix}
	1&1&0\\
	0&2&0\\
	-2&2&2
\end{bmatrix},N=\begin{bmatrix}
2&0&-1\\
2&0&-1\\
4&0&-2
\end{bmatrix}\]
6.\\
(a)首先你会发现,交换两列的初等列变换矩阵正是交换两行的初等行变换矩阵,且互为逆。然后观察到对约当块每一列向左交换,然后再每一行向下交换就变成了自己的转置,就等价于
\[J^T=E_k\cdots E_1JE_k^{-1}\cdots E_1^{-1}=(E_k\cdots E_1)J(E_k\cdots E_1)^{-1}\]
因而相似。\\
(b)
根据转置和求逆的交换性以及复方阵的约当型我们得到
\[A^T=(P^T)^{-1}J^TP^T\]
又由于每个约当块与其转置相似,我们可以得出$J,J^T$相似,因为这就相当于把一组基分成每个约当块的基然后做变换。根据相似的传递性直接得出$A$与$A^T$相似。\\
7.\\
若$A,B$可同时对角化,根据对角矩阵的交换性显然有
\[A=PD_1P^{-1},B=PD_2P^{-1}\]
\[AB=PD_1D_2P^{-1}=PD_2D_1P^{-1}=BA\]
若$AB=BA$,显然此时线性空间可以分别被$A,B$分割成各自特征子空间的直和。如果把两种分割叠加在一起,仍然是一个分割,那么我们就证完了。这相当于去证$A,B$共同的特征子空间是否线性无关。
\[\alpha_{11}+\cdots+\alpha_{1r}+\cdots+\alpha_{k1}+\cdots+\alpha_{kr}=0\]
然而根据直和的定义这显然有$\alpha_{i1}+\cdots+\alpha_{ir}=0$,进一步有$\alpha_{ij}=0$。于是此时有
\[V=(V_1\cap W_1)\oplus\cdots\oplus(V_1\cap W_r)\oplus\cdots\oplus(V_k\cap W_1)\oplus\cdots\oplus(V_k\cap W_r)\]
取这些子空间的基,显然可以组成一组$V$的基,于是同时对角化。
\end{document}
















