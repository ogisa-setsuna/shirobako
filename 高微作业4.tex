\documentclass[utf8]{ctexart}
\usepackage{graphicx}
\usepackage{amsmath}
\usepackage{amssymb}
\title{高微作业4}
\author{郑子诺,物理41}
\date{\today}
\begin{document}
\maketitle
\noindent1.\\
(1)因为
\[\sin^2(n\pi)=0\]
\[\lim\limits_{n\rightarrow\infty}(\sqrt{n^2+1}-n)=\lim\limits_{n\rightarrow\infty}\frac{1}{\sqrt{n^2+1}+n}=0\]
所以我们有
\begin{align*}
	\lim\limits_{n\rightarrow\infty}\sin^2(\pi\sqrt{n^2+1})&=\lim\limits_{n\rightarrow\infty}(\sin^2(\pi\sqrt{n^2+1})-\sin^2(n\pi))\\
	&=\lim\limits_{n\rightarrow\infty}(\sin(\sqrt{n^2+1}\pi)+\sin(n\pi))(\sin(\sqrt{n^2+1}\pi)-\sin(n\pi))\\
	&=\lim\limits_{n\rightarrow\infty}4\sin(\frac{\sqrt{n^2+1}+n}{2}\pi)\cos(\frac{\sqrt{n^2+1}-n}{2}\pi)\\
	&\sin(\frac{\sqrt{n^2+1}-n}{2}\pi)\cos(\frac{\sqrt{n^2+1}+n}{2}\pi)\\
	&\le4\lim\limits_{n\rightarrow\infty}\sin(\frac{\sqrt{n^2+1}-n}{2}\pi)\le2\pi\lim\limits_{n\rightarrow\infty}(\sqrt{n^2+1}-n)=0
\end{align*}
其中由于$\lim\limits_{n\rightarrow\infty}(\sqrt{n^2+1}-n)=0$,$\sin(\frac{\sqrt{n^2+1}-n}{2}\pi)$在$n$足够大时为正。\\
所以
\[\lim\limits_{n\rightarrow\infty}\sin^2(\pi\sqrt{n^2+1})=0\]
(2)因为
\[\sin^2((n+\frac{1}{2})\pi)=1\]
\[\lim\limits_{n\rightarrow\infty}(\sqrt{n^2+n}-n)=\lim\limits_{n\rightarrow\infty}\frac{1}{\sqrt{1+\frac{1}{n}}+1}=\frac{1}{2}\]
当$n>N,n+\frac{1}{2}-\sqrt{n^2+n}<\epsilon$时我们有(显然$n+\frac{1}{2}-\sqrt{n^2+n}>0$)
\begin{align*}
	|\sin^2((n+\frac{1}{2})\pi)-\sin^2(\pi\sqrt{n^2+n})|&=|(\sin((n+\frac{1}{2})\pi)-\sin(\pi\sqrt{n^2+n}))(\sin((n+\frac{1}{2})\pi)+\sin(\pi\sqrt{n^2+n}))|\\
	&=|4\sin(\frac{\sqrt{n^2+n}+n+\frac{1}{2}}{2}\pi)\cos(\frac{\sqrt{n^2+n}-n-\frac{1}{2}}{2}\pi)\\
	&\sin(\frac{n+\frac{1}{2}-\sqrt{n^2+n}}{2}\pi)\cos(\frac{\sqrt{n^2+n}+n+\frac{1}{2}}{2}\pi)|\\
	&\le4\sin(\frac{n+\frac{1}{2}-\sqrt{n^2+n}}{2}\pi)\le2\pi\epsilon
\end{align*}
其中由于$\lim\limits_{n\rightarrow\infty}(n+\frac{1}{2}-\sqrt{n^2+n})=0$,$\sin(\frac{n+\frac{1}{2}-\sqrt{n^2+n}}{2}\pi)$在$n$足够大时为正。\\
所以
\[\lim\limits_{n\rightarrow\infty}\sin(\pi\sqrt{n^2+n})=1\]
2.\\
(1)
\begin{align*}
	\lim\limits_{x\rightarrow+\infty}(\frac{x^2+1}{x+1}-ax-b)=\lim\limits_{x\rightarrow+\infty}((1-a)x-1-b+\frac{2}{x+1})=0
\end{align*}
显然有
\[a=1,b=-1\]
(2)
\[\lim\limits_{x\rightarrow+\infty}(\sqrt{x^2-x+1}-px-q)=\lim\limits_{x\rightarrow+\infty}(\frac{-1+\frac{1}{x}}{\sqrt{1-\frac{1}{x}+\frac{1}{x^2}}+1}+(1-p)x-q)=0\]
显然有
\[p=1,q=-1\]
3.\\
(1)因为
\[\lim\limits_{x\rightarrow0}\frac{f(x)}{x^n}=A\]
所以有
\[\lim\limits_{x\rightarrow0}f(x)=A\lim\limits_{x\rightarrow0}x^n=0\]
我们有
\[\lim\limits_{x\rightarrow0}\frac{\sqrt{1+f(x)}-1}{x^n}=\lim\limits_{x\rightarrow0}\frac{f(x)}{x^n(\sqrt{1+f(x)}+1)}=\frac{A}{2}\]
(2)因为
\[\lim\limits_{x\rightarrow0}\frac{\sin x}{x}=1\]
所以我们有
\begin{align*}
	\lim\limits_{x\rightarrow0}\frac{\sqrt{\cos x}-\sqrt{1+\sin^2x}}{x^2}&=\lim\limits_{x\rightarrow0}(\frac{\sqrt{1-2\sin^2\frac{x}{2}}-1}{x^2}+\frac{1-\sqrt{1+\sin^2x}}{x^2}\\
	&=\lim\limits_{x\rightarrow0}(-\frac{1}{2}\frac{\sin^2\frac{x}{2}}{(\frac{x}{2})^2(1+\sqrt{1-2\sin^2\frac{x}{2}})}-\frac{\sin^2x}{x^2(1+\sqrt{1+\sin^2x})})\\
	&=-\frac{3}{4}
\end{align*}
4.\\
(1)因为
\[\lim\limits_{x\rightarrow0}\frac{1-f(x)}{x^2}=A,\lim\limits_{x\rightarrow0}\frac{1-g(x)}{x^2}=B\]
所以
\[\lim\limits_{x\rightarrow0}\frac{1-f(x)}{x}=\lim\limits_{x\rightarrow0}Ax=0,\lim\limits_{x\rightarrow0}\frac{1-g(x)}{x}=0\]
所以我们有
\[\lim\limits_{x\rightarrow0}(\frac{1-f(x)}{x^2}+\frac{1-g(x)}{x^2}-\frac{1-f(x)g(x)}{x^2})=\lim\limits_{x\rightarrow0}\frac{(1-f(x))(1-g(x))}{x^2}=0\]
所以
\[\lim\limits_{x\rightarrow0}\frac{1-f(x)g(x)}{x^2}=A+B\]
(2)做归纳假设
\[\lim\limits_{x\rightarrow0}\frac{1-f_1(x)\cdots f_{k}(x)}{x^2}=A_1+A_2+\dots+A_{k}\]
$k=2$时成立,设$k=i-1$成立,下证$k=i$成立
显然令$f(x)=f_i(x),g(x)=f_1(x)\cdots f_{i-1}(x)$,由(1)得证。\\
所以
\[\lim\limits_{x\rightarrow0}\frac{1-f_1(x)\cdots f_{n}(x)}{x^2}=A_1+A_2+\dots+A_{n}\]
5.\\
由$e^x$单调性可知,任取$\epsilon>0$,令$|x|<\delta,\delta<\frac{1}{\sqrt{n}},e^n>\epsilon$则有
\[e^{-\frac{1}{x^2}}<e^{-\frac{1}{\delta^2}}<e^{-n}<\epsilon\]
因此在$x=0$处连续。\\
6.\\
显然$x=0$是一个间断点。由于$x^2-x+1,\Delta<0$,因此分母不为$0$。
若$x>1$,则有
\[\lim\limits_{n\rightarrow\infty}\frac{x^{2n+1}+1}{x^{2n+1}-x^{n+1}+x}=\lim\limits_{n\rightarrow\infty}\frac{1+\frac{1}{x^{2n+1}}}{1-\frac{1}{x^n}+\frac{1}{x^{2n}}}=1\]
若$x=1$
\[\lim\limits_{n\rightarrow\infty}\frac{x^{2n+1}+1}{x^{2n+1}-x^{n+1}+x}=1\]
若$0<x<1$
\[\lim\limits_{n\rightarrow\infty}\frac{x^{2n+1}+1}{x^{2n+1}-x^{n+1}+x}=\frac{1}{x}\]
若$-1<x<0$
\[\lim\limits_{n\rightarrow\infty}\frac{x^{2n+1}+1}{x^{2n+1}-x^{n+1}+1}=\lim\limits_{n\rightarrow\infty}\frac{(-x)^{2n+1}-1}{(-x)^{2n+1}-(-x)^{n+1}+-x}=\frac{1}{x}\]
若$x=-1$
\[\lim\limits_{n\rightarrow\infty}\frac{x^{2n+1}+1}{x^{2n+1}-x^{n+1}+1}=-1\]
若$x<-1$
\[\lim\limits_{n\rightarrow\infty}\frac{x^{2n+1}+1}{x^{2n+1}-x^{n+1}+1}=\lim\limits_{n\rightarrow\infty}\frac{1-\frac{1}{(-x)^{2n+1}}}{1-\frac{1}{(-x)^n}+\frac{1}{(-x)^{2n}}}=1\]
所以有两个间断点$x=0,-1$
7.\\
先证函数值域有界。利用二分法可知,若函数值域无界,二分取其中无界的一半,则有
\[a_{n-1}\le a_{n},b_{n-1}\ge b_n,b_n-a_n=2^{-n}(b-a)\]
且在$[a_n,b_n]$上函数值域无界。利用区间套定理知存在唯一的$c=\bigcap[a_n,b_n]$,显然$f$在$c$上有界,因此矛盾。\\
所以函数值域有界。因此函数值域存在上下确界。设下确界为$\alpha$,对于每个$\epsilon_n>0$,我们有
\[X_{\epsilon_n}={x|f(x)-\alpha<\epsilon}\subseteq[a_n,b_n]\]
显然有
\[a_{n-1}\le a_{n},b_{n-1}\ge b_n\]
令$\lim\limits_{n\rightarrow\infty}\epsilon_n=0$,根据单调序列极限定理,我们有
\[\lim\limits_{n\rightarrow\infty}a_n=A,\lim\limits_{n\rightarrow\infty}b_n=B,A\le B\]
若$A\neq B$,则闭区间$[A,B]$内存在点满足
\[f(x)-\alpha<\epsilon_1,\epsilon_2,\dots\]
这意味着$f(x)=\alpha$。\\
若$A=B$,根据区间套定理我们同样有存在一个点$c$使得$f(c)=\alpha$。上确界同理。因此函数值域取得到最大值和最小值。选取其中一对点取得最大最小值,这一对点构成一个闭区间,于是应用介值定理,我们证明了最大最小值之间每一个值都可以取到,这意味着函数值域是一个有界闭区间。
\end{document}
























