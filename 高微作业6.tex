\documentclass[utf8]{ctexart}
\usepackage{graphicx}
\usepackage{amsmath}
\usepackage{amssymb}
\title{高微作业6}
\author{郑子诺,物理41}
\date{\today}
\begin{document}
\maketitle
\noindent1.\\
(1)因为$f$处处可导,因而连续。显然$f(x)\neq0$。设$f(x)>0$,则在$x$的一个邻域内有
\[\ln|f(x)|=\ln f(x)\]
因此导函数为
\[\frac{f'(x)}{f(x)}\]
若$f(x)<0$,则在$x$的一个邻域内有
\[\ln|f(x)|=\ln(-f(x))\]
因此导函数为
\[\frac{-f'(x)}{-f(x)}=\frac{f'(x)}{f(x)}\]
综上,导函数为
\[\frac{f'(x)}{f(x)}\]
(2)显然$f(x)=1,-1$时不可导。设$f(x)\neq1,-1$,有
\[(\arcsin(f(x)))'=\frac{f'(x)}{\sqrt{1-f(x)^2}}\]
(3)显然$u(x)^{v(x)}$默认了$u(x)>0,u(x)^{v(x)}>0$,因此设$f(x)=u(x)^{v(x)}$并两边取对数有
\[\ln f(x)=v(x)\ln u(x)\]
两边求导有
\[\frac{f'(x)}{f(x)}=v'(x)\ln u(x)+\frac{u'(x)}{u(x)}v(x)\]
因此我们有
\[(u(x)^{v(x)})'=(v'(x)\ln u(x)+\frac{u'(x)}{u(x)}v(x))u(x)^{v(x)}\]
2.\\
(1)在$x_0$的一个邻域内存在一个线性变换$A:R\rightarrow R$使得
\[f(x_0+h)=f(x_0)+Ah+\alpha(h),\lim\limits_{h\rightarrow0}\frac{\alpha(h)}{h}=0\]
那么称$f$在$x_0$处可微。\\
(2)设$f$在$x_0$处可导,令
\[f(x_0+h)-f(x_0)=f'(x_0)h+\alpha(h)\]
根据导数定义有
\[\lim\limits_{h\rightarrow0}\frac{\alpha(h)}{h}=\lim\limits_{h\rightarrow0}\frac{f(x_0+h)-f(x_0)}{h}-f'(x_0)=0\]
因此可微,证毕。\\
(3)由定义可得
\[g'(0)=\lim\limits_{x\rightarrow0}\frac{g(x)-g(0)}{x}=\lim\limits_{x\rightarrow0}\frac{g(x)}{x},h'(0)=\lim\limits_{x\rightarrow0}\frac{h(x)}{x}\]
利用复合函数极限定理,绝对值函数显然连续,因此根据条件可得
\[|g'(0)|=\lim\limits_{x\rightarrow0}|\frac{g(x)}{x}|\le\lim\limits_{x\rightarrow0}|\frac{h(x)}{x}|=|h'(0)|\]
证毕。\\
3.\\
(1)\[\lim\limits_{n\rightarrow\infty}n(\ln|f(a+\frac{1}{n})|-\ln|f(a)|)=\lim\limits_{n\rightarrow\infty}\frac{\ln|f(a+\frac{1}{n})|-\ln|f(a)|}{\frac{1}{n}}=(\ln|f(a)|)'=\frac{L}{f(a)}\]
(2)显然$\dfrac{f(a+\frac{1}{n})}{f(a)}$在$n$很大时为正数,我们有
\[\lim\limits_{n\rightarrow\infty}n\ln\frac{f(a+\frac{1}{n})}{f(a)}=\lim\limits_{n\rightarrow\infty}\frac{\ln|f(a+\frac{1}{n})|-\ln|f(a)|}{\frac{1}{n}}=(\ln|f(a)|)'=\frac{L}{f(a)}\]
利用复合函数极限定理,且对数函数连续,我们有
\[\lim\limits_{n\rightarrow\infty}(\frac{f(a+\frac{1}{n})}{f(a)})^n=e^{\frac{L}{f(a)}}\]
4.\\
(1)不一定。因为$f'(x)=0$时$f^{-1}$不可导。此时有
\[\lim\limits_{y\rightarrow y_0}\frac{f^{-1}(y)-f^{-1}(y_0)}{y-y_0}=\lim\limits_{x\rightarrow x_0}\frac{x-x_0}{f(x)-f(x_0)}\rightarrow\infty\]
(2)
\[h'(x)=\frac{g'(f^{-1}(x))}{f'(x)}\]
\[h''(x)=\frac{g''(f^{-1}(x))-f''(x)g'(f^{-1}(x))}{(f'(x))^2}\]
5.\\
(1)利用复合函数高阶导数求导法则可得
\[f^{(n)}(x)=\sum_{\substack{all\,division\\i_1+\cdots+i_k=n}}g^{(k)}(h(x))h^{(i_1)}(x)\cdots h^{(i_k)}(x)\]
令$g(x)=e^x,h(x)=-\dfrac{1}{x}$,显然前者的任意阶导数仍然不变,而后者的导数是$\dfrac{1}{x}$的倍数乘上一个常数,于是我们显然有
\[f^{(n)}(x)=P_n(\frac{1}{x})e^{-\frac{1}{x}}\]
(2)利用归纳法。显然左导数任一多阶为$0$,因此以下皆讨论右导数。先求一阶导
\[f'(0)=\lim\limits_{x\rightarrow0}\frac{e^{-\frac{1}{x}}}{x}=\lim\limits_{t\rightarrow\infty}te^{-t}=0\]
下设$k-1$阶导为$0$,我们有
\[f^{(k)}(0)=\lim\limits_{x\rightarrow0}\frac{P_{k-1}(\frac{1}{x})e^{-\frac{1}{x}}}{x}=\lim\limits_{t\rightarrow\infty}tP_{k-1}(t)e^{-t}=0\]
利用归纳,我们有
\[f^{(n)}(0)=0\]
6.\\
Attention is all you need.
\[f(x)=\frac{x^2+1}{x^3-x}=\frac{1}{x-1}+\frac{1}{x+1}-\frac{1}{x}\]
因此显然有
\[f^{(n)}(x)=(-1)^nn!(\frac{1}{(x-1)^{n+1}}+\frac{1}{(x+1)^{n+1}}-\frac{1}{x^{n+1}})\]
\end{document}


















