\documentclass[utf8]{ctexart}
\usepackage{graphicx}
\usepackage{amsmath}
\usepackage{braket}
\title{费曼3作业1}
\author{郑子诺,物理41}
\date{\today}
\begin{document}
\maketitle
\noindent70.6:\\
(a):Because $L\gg d,L\gg a$,we have:
\[d\,\theta\approx m\lambda\]
\[L\theta\approx x\]
and we have de-Broglie formula:
\[\lambda=\frac{h}{p_0}\]
therefore:
\[a=\frac{L}{d}\frac{h}{p_0}\]
(b):We add the altered phase to the first equation above:
\[d\theta-\delta\phi_1+\delta\phi_2\approx m\lambda\]
and the fact that the central maximum appears when $m=0$.Then we have:
\[S=+(\delta\phi_1-\delta\phi_2)\frac{L}{d}\frac{h}{p_0}\]
(c):Because the potential changed slightly on the vertical direction,we can make Taylor expansion and reserve the first-order:
\[\frac{p^2(x)}{2m}=\frac{p^2(0)}{2m}+V(0)-V(x)\]
\[p(x)=\sqrt{p^2(0)+2m(V(0)-V(x))}\approx p(0)+\frac{m}{p(0)}(V(0)-V(x))\]
and if V(x) varies slowly,we have:
\[F=-\frac{\partial V}{\partial x}\approx -\frac{V(x)-V(0)}{x}\] 
\[p(x)=p(0)+\frac{Fx}{v}\]
(d):\\
(1):For the same reason above($L$ is very large),we can simply write:
\[\delta\phi_1-\delta\phi_2\approx (k_{up}-k_{down})L\]
and according to the de-Broglie formula:
\[p=\hbar k\]
use the result of (c) and let $x=\dfrac{d}{2}$we have:
\[\delta\phi_1-\delta\phi_2=\frac{d}{2v}\frac{F}{\hbar}L\]
(2):Use the result of (b) we have:
\[S=\frac{1}{2}F(\frac{L}{v})^2=\frac{1}{2}Ft^2\]
It's not a surprise that the classical consequence related to the central maximum of the probability amplitude,which is just corresponded to the classical limitation(the most possible situation when $\hbar\rightarrow0$).\\
70.7:\\
(a):The amplitude of "unflipped" path:
\[\braket{x|S}_1=\alpha(\braket{x|1}\braket{1|S}+\braket{x|2}\braket{2|S})\]
The amplitude of "flipped" path:
\[\braket{x|S}_2=\beta(\braket{x|1}\braket{1|S}+\braket{x|2}\braket{2|S})\]
Therefore the distribution $P$ is:
\[P=(|\alpha|^2+|\beta|^2)|\braket{x|1}\braket{1|S}+\braket{x|2}\braket{2|S}|^2\]
(b):Obviously,the answer is the same as (a):
\[P=(|\alpha|^2+|\beta|^2)|\braket{x|1}\braket{1|S}+\braket{x|2}\braket{2|S}|^2\]
(c):For every electron,no matter what spin it has,the distribution is the same,so the answer is the same:
\[P=(|\alpha|^2+|\beta|^2)|\braket{x|1}\braket{1|S}+\braket{x|2}\braket{2|S}|^2\]
70.9:\\
(a):It's important that the photons are identical,so that the amplitude of $p_{12}$ should contain two case:
\[\braket{ab|\psi_{12}}=\braket{a|A}\braket{b|B}+\braket{b|A}\braket{a|B}\]
and use symmetry:
\[\braket{ab|\psi_{12}}=c^2(\exp(2i\alpha_1)+\exp(2i\alpha_2))\]
then we have:
\[p_{12}=|\braket{ab|\psi_{12}}|^2=|c|^4(2+2\cos2k(R_2-R_1))\]
(b):Let $\theta=\dfrac{D}{2R}$,we have:
\[d\,\theta\approx R_2-R_1\]
\[p_{12}\propto2+2\cos\frac{kD}{R}d\]
Hence we can measure how $p_{12}$ changed when $d$ varies,which is just like the distribution of interference.If we have measured the interval $\Delta d$ between two maximum,then we have:
\[D=\frac{R}{\Delta d}\lambda\qquad\text{where }\lambda\text{ is the wavelength of the light}\]
\end{document}




















