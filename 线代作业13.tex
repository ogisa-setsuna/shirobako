\documentclass[utf8]{ctexart}
\usepackage{graphicx}
\usepackage{amsmath}
\usepackage{amssymb}
\usepackage{braket}
\title{线代作业13}
\author{郑子诺,物理41}
\date{\today}
\begin{document}
\maketitle
\noindent1.\\
(a)\[\begin{bmatrix}
	-\dfrac{1}{\sqrt{2}}&\dfrac{1}{\sqrt{2}}\\[8pt]
	\dfrac{1}{\sqrt{2}}&\dfrac{1}{\sqrt{2}}
\end{bmatrix}\begin{bmatrix}
\sqrt{2}&\sqrt{2}\\
0&4\sqrt{2}
\end{bmatrix}\]
(b)
\[\begin{bmatrix}
	\dfrac{1}{\sqrt{2}}&\dfrac{1}{\sqrt{6}}&-\dfrac{1}{\sqrt{3}}\\[8pt]
	\dfrac{1}{\sqrt{2}}&-\dfrac{1}{\sqrt{6}}&\dfrac{1}{\sqrt{3}}\\[8pt]
	0&\dfrac{2}{\sqrt{6}}&\dfrac{1}{\sqrt{3}}
\end{bmatrix}\begin{bmatrix}
\sqrt{2}&\dfrac{1}{\sqrt{2}}&\dfrac{1}{\sqrt{2}}\\[8pt]
0&\dfrac{\sqrt{6}}{2}&\dfrac{1}{\sqrt{6}}\\[8pt]
0&0&\dfrac{2}{\sqrt{3}}
\end{bmatrix}\]
2.\\
我们有
\[A^TAx=A^Ty\rightarrow a=-\frac{36}{35},b=\frac{61}{35}\]
3.\\
(a)根据向量范数的三角不等式,我们有
\[\Vert(A+B)x\Vert\le\Vert Ax\Vert+\Vert Bx\Vert\]
因此
\[\Vert A+B\Vert=\max_{|x|=1}\Vert (A+B)x\Vert\le\max_{|x|=1}(\Vert Ax\Vert+\Vert Bx\Vert)\le\Vert A\Vert+\Vert B\Vert\]
(b)由于$Bx$的值域小于等于全空间,因此有
\[\frac{\Vert ABx\Vert}{\Vert Bx\Vert}\le\Vert A\Vert\]
于是
\[\Vert AB\Vert=\max_{|x|=1}\Vert ABx\Vert\le\max_{|x|=1}\Vert A\Vert\Vert Bx\Vert=\Vert A\Vert\Vert B\Vert\]
4.\\
由于正定方阵可对角化,因此我们有
\[e^{-tA}=e^{-tU\Sigma U^{-1}}=\sum_{k=0}^\infty\frac{(-tU\Sigma U^{-1})^k}{k!}=U\sum_{k=0}^\infty\frac{-t\Sigma^k}{k!}U^{-1}=Ue^{-t\Sigma}U^{-1}\]
鉴于正定方阵特征值为正,因而我们有
\[\lim_{t\rightarrow+\infty}e^{-tA}=\lim_{t\rightarrow+\infty}U\begin{bmatrix}
	e^{-t\lambda_1}&&&\\
	&e^{-t\lambda_2}&&\\
	&&\ddots&\\
	&&&e^{-t\lambda_k}
\end{bmatrix}U^{-1}=0\]
5.\\
显然其解为
\[Y=e^{tA}Y_0,A=\begin{bmatrix}
	-4&1\\
	-2&-1
\end{bmatrix}\]
显然$A$的特征值为$-2,-3$,于是直接由上题知
\[\lim_{t\rightarrow+\infty}x(t)=\lim_{t\rightarrow+\infty}y(t)=0\]
\end{document}











