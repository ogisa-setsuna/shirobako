\documentclass[utf8]{ctexart}
\usepackage{graphicx}
\usepackage{amsmath}
\title{热学笔记}
\author{haruki}
\date{\today}
\begin{document}
\maketitle
\tableofcontents
\section{旋转参考系下的麦克斯韦分布律}
\noindent 该笔记灵感源于竞赛题。我们经常会碰到这样的情况:一个圆筒以$\boldsymbol{\omega}$的角速度旋转,使其中空气以相同角速度整体旋转,相对静止,此时我们在旋转参考系中
仍然使用麦克斯韦-玻尔兹曼分布律。这真的合理吗?难道真的只是加上离心势能就可以了吗?\\
问题是显而易见的,与惯性系不同,旋转参考系下会受到两个惯性力(匀速旋转),一个是离心力,另一个是科里奥利力,前者可以当做势能处理,但后者也起着关键
的作用,因而不可忽视。由此观之,我们并没有直接的理由认为麦克斯韦-玻尔兹曼分布律仍成立。\\
为了解决这个问题,我们需要回到统计力学的最开始处。固然,哈密顿正则方程才是统计力学源点,因此,寻找旋转系的哈密顿量是首要的任务。可以看出,此时的哈密顿量
不再是$\frac{1}{2}m\boldsymbol{v^2}-\frac{1}{2}m\boldsymbol{\omega} ^{2}r^2$($\boldsymbol{v}$为旋转系中速度),将上式带入正则方程得不到正确的受力方程。让我们从改写拉格朗日量开始。\\
初始的拉氏量为$L=\frac{1}{2}m\boldsymbol{v_0^{2}}-U$,转换至平动非惯性系有:$\boldsymbol{v_0}=\boldsymbol{v^{'}}+\boldsymbol{u}(t)$,其中$\frac{d\boldsymbol{u}}{dt}=\boldsymbol{a}(t)$,于是
$$
L=\frac{1}{2}m\boldsymbol{v^{'2}}+m\boldsymbol{v^{'}}\cdot\boldsymbol{u}(t)+\frac{1}{2}m\boldsymbol{u^{2}}(t)-U
$$
因为$\boldsymbol{v^{'}}=\frac{d\boldsymbol{r^{'}}}{dt}$,所以有
$$
m\boldsymbol{v^{'}}\cdot\boldsymbol{u}(t)=\frac{d}{dt}(m\boldsymbol{r^{'}}\cdot\boldsymbol{u}(t))-m\boldsymbol{r^{'}}\cdot\frac{d\boldsymbol{u}}{dt}
$$
根据拉格朗日量的性质,可以增减一项时间的全导数。因此此时拉氏量为:
$$
L=\frac{1}{2}m\boldsymbol{v^{'2}}-m\boldsymbol{r^{'}}\cdot\boldsymbol{a}(t)-U
$$
我们再换到旋转系中,有$\boldsymbol{v^{'}}=\boldsymbol{v}+\boldsymbol{\omega}\times\boldsymbol{r}$,且$\boldsymbol{r}=\boldsymbol{r^{'}}$,所以:
$$
L=\frac{1}{2}m\boldsymbol{v^{2}}+m\boldsymbol{v}\cdot(\boldsymbol{\omega}\times\boldsymbol{r})+\frac{1}{2}m(\boldsymbol{\boldsymbol{\omega}\times\boldsymbol{r}})^2-m\boldsymbol{r}\cdot\boldsymbol{a}-U
$$
这就是旋转参考系的拉格朗日量。不难验证受力方程的正确性。\\
现在回归正题,另$\boldsymbol{u}(t)=0,\boldsymbol{\omega}=const$,得到:
$$
L=\frac{1}{2}m\boldsymbol{v^{2}}+m\boldsymbol{v}\cdot(\boldsymbol{\omega}\times\boldsymbol{r})+\frac{1}{2}m(\boldsymbol{\boldsymbol{\omega}\times\boldsymbol{r}})^2-U
$$
易得广义动量为:
$$
\boldsymbol{p}=\frac{\partial L}{\partial \boldsymbol{v}}=m\boldsymbol{v}+m\boldsymbol{\omega}\times\boldsymbol{r}
$$
代入$H=\boldsymbol{p}\cdot\boldsymbol{v}-L$得:
$$
H=\frac{1}{2}m\boldsymbol{v^2}-\frac{1}{2}m\boldsymbol{\omega} ^{2}r^2-U
$$
得到了一开始否定的哈密顿量。显然,此时成立的原因是广义动量$\boldsymbol{p}$的改变。\\
但是到这里为止证明并没有结束。众所周知,经典近似下的统计力学需要对相空间进行积分,而此时的微元为$d^3\boldsymbol{p}d^3\boldsymbol{r}$,由于广义动量$\boldsymbol{p}$不再等于$m\boldsymbol{v}$,因此为了得出正确的分布律,还需证明微元$d^3\boldsymbol{p}d^3\boldsymbol{r}=d^3m\boldsymbol{v}d^3\boldsymbol{r}$。\\
雅克比行列式为:
$$
\begin{vmatrix}
	1  &0  &0  &0&-\boldsymbol{\omega_{z}}&\boldsymbol{\omega_{y}}\\
	0  &1  &0  &\boldsymbol{\omega_{z}}&0&-\boldsymbol{\omega_{x}}\\
	0  &0  &1  &-\boldsymbol{\omega_{y}}&\boldsymbol{\omega_{x}}&0\\
	0  &0  &0  &1&0&0\\
	0  &0  &0  &0&1&0\\
	0  &0  &0  &0&0&1\\
\end{vmatrix}
$$
显然为$1$。因此可以直接换元。于是我们就得到了梦寐以求的麦克斯韦-玻尔兹曼分布律,以后使用公式的时候也可以安心了呢(\^{}o\^{})
\begin{center}
	\includegraphics[width=\textwidth]{shinobu.png}
	\textbf{小忍}
\end{center}
\section{表面张力与接触角}
\noindent 这一段笔记源于学习刘玉鑫《热学》第六章中关于接触角讨论的一些困惑。首先,原书中的图显然是有问题的,那个受力分析显然有误。同时,关于多个表面张力系数的出现也令人疑惑。因此首先来考察一下表面张力的起因。\\
一般认为,表面张力来源于表面分子的相互作用能增大,因此有缩小表面使得能量变小的趋势,从而产生了表面张力。然而这样的理解缺乏直接的物理图像,因此下面将阐述我自己的理解。\\
首先,近似的说,由玻尔兹曼分布律可知表面分子的数密度由于势能增大而显著减少,距离相对增大根据Lennard-Jones势,分子间作用力呈吸引力,从而有表面张力。从这个角度我们来分析固体表面一个液滴在与固体接触的界面上产生的力。为简单起见,固体分子视作固定不动,忽略空气作用。\\
由于固体分子对液体分子的吸引作用,界面的液体分子数密度将显著增加,距离相对缩小,由Lennard-Jones势,分子间作用力呈排斥力,从而液体与固体的接触面有扩大趋势。但是以上讨论忽略了另一个作用,那便是液体分子内部对界面层分子的吸引力。前者称作\textbf{吸附力},后者称作\textbf{内聚力}。因而在综合考虑了吸附力和内聚力的情况下,界面层的液体分子数密度可能增加也可能减少,产生的现象便是\textbf{浸润}与\textbf{不浸润}。无论是哪一种情况,我们可以看到,类似于表面张力,界面层边缘的分子将受到正比于长度的作用力,系数记作$\sigma_2$,与表面张力系数$\sigma_1$对应。\\
为了研究接触角,我们要着眼于界面层边界的分子,注意到它们同时也是液体表面层的分子,因而收到表面张力和界面张力两个力,此时为了受力平衡理应受到一个正比于长度的吸附力,相比之下液滴底面似乎并没有受到什么力。这其实是因为界面层底层的分子已经落入固体分子束缚态,在受到外力使其脱离时会产生很强的吸附力,但是平时我们考察的液滴不包括界面层,因此在计算底面压力时不存在这一项。此时我们考察的是界面层边界分子,由于受到表面张力向外拉扯,自然会受到吸附力阻止其脱离。于是由受力平衡得:
$$
\cos\theta=\frac{\sigma_2}{\sigma_1},\sigma_2\text{在排斥力时取正}
$$
显然,当$\sigma_2$是负的时候其绝对值最大也不可能超过$\sigma_1$,因为此时固体分子吸引力趋于$0$,界面层与表面层无异。当$\sigma_2$是正的时候其大小可能超过$\sigma_1$,因而此时无法形成成型的液滴,完全浸润。\\
以上讨论完毕。
\begin{center}
	\includegraphics[width=\textwidth]{As109shinobu.jpg}
	\textbf{小忍}
\end{center}
\end{document}


