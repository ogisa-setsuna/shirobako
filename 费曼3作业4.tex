\documentclass[utf8]{ctexart}
\usepackage{graphicx}
\usepackage{amsmath}
\usepackage{amssymb}
\usepackage{braket}
\title{费曼3作业4}
\author{郑子诺,物理41}
\date{\today}
\begin{document}
\maketitle
74.1\\
(a)
\[V=+\mu B,0,-\mu B\]
\[\Delta(\frac{p_0^2}{2M})\approx\frac{p_0\Delta p_0}{M}=-\Delta V\]
\[\Delta\phi=\Delta p_0L=-\frac{ML}{p_0}\Delta V=p_y\Delta z\]
\[\therefore \delta\theta\approx\frac{p_y}{p_0}=-\frac{ML}{p_0^2}\frac{\Delta V}{\Delta z}\approx-\frac{ML}{p_0^2}\frac{\partial V}{\partial z}\]
For different states,we have
\[\delta\theta=-\frac{ML}{p_0^2}\frac{\mu\partial B}{\partial z},0,+\frac{ML}{p_0^2}\frac{\mu\partial B}{\partial z}\]
(b)
\begin{align}
	\ket{+x(t)}&=\braket{+z|+x}\ket{+z(t)}+\braket{0z|+x}\ket{0z(t)}+\braket{-z|+x}\ket{-z(t)}\notag\\
	&=\frac{1}{2}e^{-i\frac{\mu B}{\hbar}t}\ket{+z}-\frac{1}{\sqrt{2}}\ket{0z}+\frac{1}{2}e^{i\frac{\mu B}{\hbar}t}\ket{-z}\notag
\end{align}
\[\ket{0x(t)}=\frac{1}{\sqrt{2}}e^{-i\frac{\mu B}{\hbar}t}\ket{+z}-\frac{1}{\sqrt{2}}e^{i\frac{\mu B}{\hbar}t}\ket{-z}\]
\[\ket{-x(t)}=\frac{1}{2}e^{-i\frac{\mu B}{\hbar}t}\ket{+z}+\frac{1}{\sqrt{2}}\ket{0z}+\frac{1}{2}e^{i\frac{\mu B}{\hbar}t}\ket{-z}\]
\begin{align}
	\braket{+x|+x(t)}&=\braket{+x|+z(t)}\braket{+z|+x}+\braket{+x|0z(t)}\braket{0z|+x}+\braket{+x|-z(t)}\braket{-z|+x}\notag\\
	&=\frac{1}{2}(\cos(\frac{\mu B}{\hbar}t)+1)\notag
\end{align}
\[\braket{0x|0x(t)}=\cos(\frac{\mu B}{\hbar}t)\]
\[\braket{-x|-x(t)}=\frac{1}{2}(\cos(\frac{\mu B}{\hbar}t)+1)\]
(c)
Method 1:Measure the deflection of a beam of particles that passed through an inhomogeneous magnetic field,then we have
\[\mu=\frac{p_0^2}{ML\frac{\partial B}{\partial z}}\delta\theta\]
Method 2:Measure the oscillation angular frequency of the probability of detecting a particle that has a spin on $+x$ direction,which is put in a uniform magnetic field with $+z$ direction.Then we have
\[\mu=\frac{\hbar}{B}\omega\]
73.3\\
(a)
\begin{align*}
	N&=N_0|\braket{+U|+T}\braket{+T|-S}+\braket{+U|-T}\braket{-T|-S}|^2\\
	&=N_0|\braket{+U|-S}|^2
\end{align*}
(b)
\[\begin{bmatrix}
	0&1\\
	-1&0
\end{bmatrix}
\begin{bmatrix}
	0\\
	1
\end{bmatrix}
=
\begin{bmatrix}
	1\\
	0
\end{bmatrix}\]
\[\therefore\braket{+T|-S}=1,\braket{-T|-S}=0\]
(c)
\[\begin{bmatrix}
	C_1'\\
	C_2'
\end{bmatrix}
=R_y(-\frac{\pi}{2})R_z(\theta)R_y(\frac{\pi}{2})\begin{bmatrix}
	C_1\\
	C_2
\end{bmatrix}
=\begin{bmatrix}
	\cos\frac{\theta}{2}&i\sin\frac{\theta}{2}\\
	i\sin\frac{\theta}{2}&\cos\frac{\theta}{2}
\end{bmatrix}\begin{bmatrix}
C_1\\
C_2
\end{bmatrix}\]
\[\therefore\braket{+U|-S}=i\sin\frac{\theta}{2}\]
(d)
\[\theta=0,\braket{+U|-S}=0\]
\[\theta=\pi,\braket{+U|-S}=i\]
(e)
Because $U$ need a rotation along $z$-axis to become $T$.
\[\begin{bmatrix}
	-i&0\\
	0&i
\end{bmatrix}\begin{bmatrix}
	0&i\\
	i&0
\end{bmatrix}=\begin{bmatrix}
0&1\\
-1&0
\end{bmatrix}\]
73.6\\
(a)
\[P_{+z}=|\sqrt{1+\frac{v}{c}}\sin\frac{\theta}{2}|^2=(1+\frac{v}{c})\sin^2\frac{\theta}{2}\]
\[P_{-z}=(1-\frac{v}{c})\cos^2\frac{\theta}{2}\]
(b)
\begin{align*}
	P_{+x}&=|\frac{1}{\sqrt{2}}\sqrt{1+\frac{v}{c}}\sin\frac{\theta}{2}+\frac{1}{\sqrt{2}}\sqrt{1-\frac{v}{c}}\cos\frac{\theta}{2}|^2\\
	&=\frac{1}{2}(1-\frac{v}{c}\cos\theta+\sqrt{1-\frac{v^2}{c^2}}\sin\theta)
\end{align*}
\[P_{-x}=\frac{1}{2}(1-\frac{v}{c}\cos\theta-\sqrt{1-\frac{v^2}{c^2}}\sin\theta)\]
(c)
\begin{align*}
	P_{+y}&=|\frac{1}{\sqrt{2}}\sqrt{1+\frac{v}{c}}\sin\frac{\theta}{2}-i\frac{1}{\sqrt{2}}\sqrt{1-\frac{v}{c}}\cos\frac{\theta}{2}|^2\\
	&=\frac{1}{2}(1-\frac{v}{c}\cos\theta)
\end{align*}
\[P_{-y}=\frac{1}{2}(1-\frac{v}{c}\cos\theta)\]
(d)
\[P_{+z}=\int_{0}^{\pi}\frac{1}{2}\sin\theta(1+\frac{v}{c})\sin^2\frac{\theta}{2}d\theta=\frac{1}{2}(1+\frac{v}{c})\]
\[P_{-z}=\frac{1}{2}(1-\frac{v}{c})\]
\end{document}















