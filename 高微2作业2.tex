\documentclass[utf8]{ctexart}
\usepackage{graphicx}
\usepackage{amsmath}
\usepackage{amssymb}
\title{高微2作业2}
\author{郑子诺,物理41}
\date{\today}
\begin{document}
\maketitle
\noindent1.\\
(1)鉴于
\[\limsup\sqrt[n]{|a_n|}=\limsup\sqrt[n]{\frac{1}{n}}=1\]
我们有
\[R=1\]
当$x=1$时,利用莱布尼茨判别法知该交错级数收敛。当$x=-1$时,此恰为调和级数,发散。因此收敛域为$(-1,1]$。\\
(2)设和函数为$f(x)$,我们有
\[f'(x)=\sum_{n=0}^{+\infty}(-1)^nx^n=\frac{1}{1+x},|x|<1\]
积分并利用$f(0)=0$得到
\[f(x)=\ln(1+x),|x|<1\]
(3)由于当$x=1$时,利用莱布尼茨判别法知该交错级数收敛,再利用阿贝尔定理知和函数在$x=1$处连续,因此
\[\sum_{n=1}^{\infty}\frac{(-1)^{n-1}}{n}=\ln 2\]
2.\\
(1)对$\arctan x$求导得$\dfrac{1}{1+x^2}$,后者展开为
\[\frac{1}{1+x^2}=1-x^2+x^4-x^6+\cdots,|x|<1\]
逐项积分得
\[\arctan x=\sum_{n=0}^{\infty}(-1)^n\frac{x^{2n+1}}{2n+1},|x|<1\]
(2)鉴于积分不改变收敛半径,收敛半径仍为$1$。\\
(3)当$x=1$时,根据莱布尼茨判别法,该交错级数收敛,再根据阿贝尔定理知和函数在$x=1$处连续,因此
\[\frac{\pi}{4}=\sum_{n=0}^{\infty}\frac{(-1)^n}{2n+1}\]
3.\\
(1)收敛域是显然的,为$(1,+\infty)$。\\
(2)是。对于$x\in[b,+\infty)$,其中$b>1$,我们有
\[\frac{1}{n^x}\le\frac{1}{n^b}\]
而后者显然收敛,因此该级数在$[b,+\infty)$上一致收敛。对于任一$x\in X$,取$1<b<x$,利用上述结论容易知$\zeta(x)$在$X$上连续。\\
(3)对级数逐项求导得到级数
\[\sum_{n=1}^{\infty}\frac{-\ln n}{n^x}\]
对于$x\in[b,+\infty)$,其中$b>1$,我们有
\[\frac{\ln n}{n^x}\le\frac{\ln n}{n^b}\]
对后者使用比较判别法,取级数$\dfrac{1}{n^{1+\epsilon}}$,其中$\epsilon<b-1$,我们有
\[\lim_{n\rightarrow+\infty}\frac{\ln n}{n^{b-1-\epsilon}}=0\]
因此该级数收敛,因而求导后的级数在$[b,+\infty)$上一致收敛。于是对于任一$x\in X$,取$1<b<x$,根据一致收敛的性质得
\[\zeta'(x)=\sum_{n=1}^{+\infty}\frac{-\ln n}{n^x}\]
4.\\
(1)利用比值判别法
\[\lim_{n\rightarrow+\infty}\frac{a_{n+1}}{a_n}=\frac{2n+1}{(n+1)^2}=4\]
因此收敛半径为$\dfrac{1}{4}$。\\
(2)
\[\frac{1}{\sqrt{1-x}}=1+\sum_{n=1}^{+\infty}\frac{(2n-1)!!}{n!2^n}x^n,|x|<1\]
(3)利用前一小题结论,由于$\arcsin x$导数为$\dfrac{1}{\sqrt{1-x^2}}$,我们有
\[\frac{1}{\sqrt{1-x^2}}=1+\sum_{n=1}^{+\infty}\frac{(2n-1)!!}{n!2^n}x^{2n},|x|<1\]
利用幂函数性质逐项积分得
\[\arcsin x=x+\sum_{n=1}^{+\infty}\frac{(2n-1)!!}{n!2^n}\frac{x^{2n+1}}{2n+1},|x|<1\]
5.\\
(1)鉴于
\[|f_n(x)-f_n(a)|=|f'(\xi)(x-a)|\le M_n(x-a)\]
后者由题设知收敛。因此级数$\sum\limits_{n=1}^{+\infty}(f_n(x)-f_n(a))$绝对收敛,于是由题设知级数$\sum\limits_{n=1}^{+\infty}f_n(x)$逐点收敛。\\
(2)鉴于题设,$\sum\limits_{n=1}^{+\infty}f'_n(x)$显然在$(a,b)$上一致收敛,设其和函数为$g(x)$,根据$f'_n(x)$连续我们有
\[f_n(x)=f_n(x_0)+\int_{x_0}^{x}f'_n\]
其中$x,x_0\in(a,b)$。左右两边求和取极限并利用一致连续的性质得
\[S(x)=S(x_0)+\int_{x_0}^{x}g\]
显然$S(x)$在$(a,b)$上处处可导。\\
6.\\
首先由题设得
\[|\sum_{i=m}^{n}a_i(x)b_i(x)|\le(|a_n(x)|+|a_m(x)-a_n(x)|)M\]
再由$a_n(x)$一致收敛至$0$,存在$N$使得$\forall n>N,|a_n|<\dfrac{\epsilon}{3M}$。因此对于任一$m,n>N$我们有
\[|\sum_{i=m}^{n}a_i(x)b_i(x)|\le\epsilon\]
由一致收敛的柯西判别法知该级数一致收敛。\\
7.\\
将级数写成
\[\sum_{n=1}^{+\infty}a_n(x)b_n(x)=\sum_{n=1}^{+\infty}(a_n(x)-a(x))b_n(x)+\sum_{n=1}^{+\infty}a(x)b_n(x)\]
其中$a(x)$为$a_n(x)$的逐点极限。前一项由上题知一致收敛,后一项显然也一致收敛,因为存在$N$使得$m,n>N,|b_m(x)+\cdots+b_n(x)|<\dfrac{\epsilon}{K}$,于是
\[|a(x)b_m(x)+\cdots+a(x)b_n(x)|<\epsilon\]
因此原级数一致收敛。\\
8.\\
当$x\in[0,x_0]$时,我们有
\[a_nx^n=a_n\frac{x^n}{x_0^n}x_0^n\le a_nx_0^n\]
而后者收敛。因此该级数在$[0,x_0]$上一致收敛。
\end{document}














