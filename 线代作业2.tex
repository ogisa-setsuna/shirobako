\documentclass[utf8]{ctexart}
\usepackage{graphicx}
\usepackage{amsmath}
\title{线性代数第二次作业}
\author{郑子诺,物理41}
\date{\today}
\begin{document}
\maketitle
\noindent Q1.
\[
\begin{vmatrix}
	1&1&1&1\\
	0&1&1&1\\
	0&0&1&1\\
	0&0&0&1\\
\end{vmatrix}
=1
\]
显然线性无关,又由于$R^4$为四维线性空间,所以构成一组基。\\
Q2.$z=x+2y$,所以令$x=1,y=0$和$x=0,y=1$得:
\[\mathbf{v_1}=(1,0,1),\mathbf{v_2}=(0,1,2)\]显然线性无关。并且每一个解可以表示为:
\[\mathbf{v}=(x,y,x+2y)=x\mathbf{v_1}+y\mathbf{v_2}\]因此子空间维数是$2$,$\mathbf{v_1},\mathbf{v_2}$构成一组基。\\
Q3.易解得$x_3=4x_1+x_2,x_4=3x_1$所以令$x_1=1,x_2=0$和$x_1=0,x_2=1$得:
\[\mathbf{v_1}=(1,0,4,3),\mathbf{v_2}=(0,1,1,0)\]显然线性无关。并且每一个解可以表示为:
\[\mathbf{v}=(x_1,x_2,4x_1+x_2,3x_1)=x_1\mathbf{v_1}+x_2\mathbf{v_2}\]因此子空间维数是$2$,$\mathbf{v_1},\mathbf{v_2}$构成一组基。\\
Q4.显然一组基$\{A_{ij}\}$为:
\[a_{mn}=
\begin{cases}
	0 &\text{if } m,n\neq i,j,\\
	1 &\text{if } m,n=i,j\\
\end{cases}\]
线性无关是显然的,并且每一个矩阵为:
\[B=b_{ij}A_{ij}\]因此维数为$mn$。\\
Q5.\\
(a)因为
\[
\begin{vmatrix}
	1&2&3\\
	2&1&3\\
	3&0&3\\
\end{vmatrix}
=0
\]
而$\mathbf{v_1},\mathbf{v_2}$显然线性无关,所以秩为$2$,一个极大线性无关组为$\mathbf{v_1},\mathbf{v_2}$。\\
(b)因为
\[\mathbf{v_3}=\mathbf{v_1}+\mathbf{v_2},\mathbf{v_4}=2\mathbf{v_1}+\mathbf{v_2}\]
显然$\mathbf{v_1},\mathbf{v_2}$线性无关。所以秩为$2$,一个极大线性无关组为$\mathbf{v_1},\mathbf{v_2}$。\\
Q6.因为$\alpha_1,\alpha_2,\alpha_3$线性无关,所以$\alpha_1,\alpha_2$线性无关,又由于$\beta$不能由$\alpha_1,\alpha_2$线性表出,若三者线性相关,将会得出$\beta$系数为$0$,从而得出$\alpha_1,\alpha_2$线性相关的矛盾,因此三者线性无关。
又由于$\beta$可由$\alpha_1,\alpha_2,\alpha_3$线性表出,所以$\{\beta,\alpha_1,\alpha_2,\alpha_3\}$秩为$3$,因此$\{\beta,\alpha_1,\alpha_2\}$为最大线性无关组。\\
Q7.由课上所证的替代定理可知,向量组$\mathbf{v}$的极大线性无关组可以讲$\mathbf{u}$的极大线性无关组替代,因此前者个数一定小于等于后者,于是得证。\\
Q8.由Q7得到两个相反的不等式,因此得证。\\
Q9.\\
(a)\[
\begin{bmatrix}
	1&2\\
	3&-1\\
\end{bmatrix}
\]
(b)\[
\begin{bmatrix}
	1&1&0\\
	0&1&-1\\
\end{bmatrix}
\]
(c)\[
\begin{bmatrix}
	1&0\\
	1&1\\
	1&-1\\
\end{bmatrix}
\]
(d)\[
\begin{bmatrix}
	1&0\\
	0&-1\\
\end{bmatrix}
\]
(e)\[
\begin{bmatrix}
	0&1\\
	1&0\\
\end{bmatrix}
\]
(f)\[
\begin{bmatrix}
	\frac{1}{\sqrt{2}}&-\frac{1}{\sqrt{2}}\\
	-\frac{1}{\sqrt{2}}&-\frac{1}{\sqrt{2}}\\
\end{bmatrix}
\]
\end{document}


















