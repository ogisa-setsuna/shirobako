\documentclass[utf8]{ctexart}
\usepackage{graphicx}
\usepackage{amsmath}
\usepackage{amssymb}
\usepackage{siunitx}
\makeatletter
\newcommand{\rmnum}[1]{\romannumeral #1}
\newcommand{\Rmnum}[1]{\expandafter\@slowromancap\romannumeral #1@}
\makeatother
\makeatletter
\newsavebox\myboxA
\newsavebox\myboxB
\newlength\mylenA
\newcommand*\xoverline[2][0.75]{%
	\sbox{\myboxA}{$\m@th#2$}%
	\setbox\myboxB\null% Phantom box
	\ht\myboxB=\ht\myboxA%
	\dp\myboxB=\dp\myboxA%
	\wd\myboxB=#1\wd\myboxA% Scale phantom
	\sbox\myboxB{$\m@th\overline{\copy\myboxB}$}% Overlined phantom
	\setlength\mylenA{\the\wd\myboxA}% calc width diff
	\addtolength\mylenA{-\the\wd\myboxB}%
	\ifdim\wd\myboxB<\wd\myboxA%
	\rlap{\hskip 0.5\mylenA\usebox\myboxB}{\usebox\myboxA}%
	\else
	\hskip -0.5\mylenA\rlap{\usebox\myboxA}{\hskip 0.5\mylenA\usebox\myboxB}%
	\fi}
\makeatother
\newcommand{\bm}[1]{\boldsymbol{#1}}
\title{统计力学作业7}
\author{郑子诺,物理41}
\date{\today}
\begin{document}
\maketitle
\noindent7.11\\
类似三维的麦克斯韦分布,可以直接写出速度分布为
\[f(\bm{v})=\left(\frac{m}{2\pi kT}\right)e^{-\frac{m(v_x^2+v_y^2+v_z^2)}{2kT}}\]
速率分布为
\[f(v)=2\pi v\left(\frac{m}{2\pi kT}\right)e^{-\frac{mv^2}{2kT}}\]
\[\xoverline{v}=\int_{0}^{+\infty}2\pi v^2\left(\frac{m}{2\pi kT}\right)e^{-\frac{mv^2}{2kT}}\mathrm{d}v=\sqrt{\frac{\pi kT}{2m}}\]
\[\frac{\mathrm{d}f(v)}{\mathrm{d}v}=0\rightarrow v_m=\sqrt{\frac{kT}{m}}\]
\[v_s^2=\int_{0}^{+\infty}2\pi v^3\left(\frac{m}{2\pi kT}\right)e^{-\frac{mv^2}{2kT}}\mathrm{d}v=\sqrt{\frac{2kT}{m}}\rightarrow v_s=\sqrt{\frac{2kT}{m}}\]
\setlength{\arraycolsep}{8pt}
7.12\\
设$\bm{v_r}=\bm{v_2}-\bm{v_1},v_c=\dfrac{\bm{v_1}+\bm{v_2}}{2}$,我们有
\[\bm{v_1}=\bm{v_c}-\frac{\bm{v_r}}{2},\bm{v_2}=\bm{v_c}+\frac{\bm{v_r}}{2}\]
换元的雅可比行列式为
\begin{align*}
	J&=\begin{bmatrix}
		1&0&0&-\dfrac{1}{2}&0&0\\[8pt]
		0&1&0&0&-\dfrac{1}{2}&0\\[8pt]
		0&0&1&0&0&-\dfrac{1}{2}\\[8pt]
		1&0&0&\dfrac{1}{2}&0&0\\[8pt]
		0&1&0&0&\dfrac{1}{2}&0\\[8pt]
		0&0&1&0&0&\dfrac{1}{2}\\[8pt]
	\end{bmatrix}\\
	&=\begin{bmatrix}
		2&0&0&0&0&0\\[8pt]
		0&2&0&0&0&0\\[8pt]
		0&0&2&0&0&0\\[8pt]
		1&0&0&\dfrac{1}{2}&0&0\\[8pt]
		0&1&0&0&\dfrac{1}{2}&0\\[8pt]
		0&0&1&0&0&\dfrac{1}{2}\\[8pt]
	\end{bmatrix}\\
	&=\det\begin{bmatrix}
		2&0&0\\[8pt]
		0&2&0\\[8pt]
		0&0&2\\[8pt]
	\end{bmatrix}\det\begin{bmatrix}
	\dfrac{1}{2}&0&0\\[8pt]
	0&\dfrac{1}{2}&0\\[8pt]
	0&0&\dfrac{1}{2}\\[8pt]
	\end{bmatrix}=1 
\end{align*}
因此我们有
\[f(\bm{v_r},\bm{v_c})=J\left(\frac{m}{2\pi kT}\right)^3e^{-\frac{mv_1^2}{2kT}}e^{-\frac{mv_2^2}{2kT}}=\left(\frac{2m}{2\pi kT}\right)^\frac{3}{2}e^{-\frac{2mv_c^2}{2kT}}\left(\frac{m}{4\pi kT}\right)^\frac{3}{2}e^{-\frac{mv_r^2}{4kT}}\]
因此
\[\xoverline{v_r}=4\sqrt{\frac{kT}{\pi m}}\]
7.17\\
此时的配分函数为
\[Z_1=\frac{S}{h^3}\int e^{-\beta\epsilon(\bm{p})}\mathrm{d}\bm{p}\int_{0}^{H}e^{-\beta mgz}\mathrm{d}z=Z_0\frac{1-e^{-\beta mgH}}{mg\beta}\]
因此
\[U=-\frac{N\partial\ln Z_1}{\partial \beta}=U_0+NkT-\frac{NmgH}{e^{\frac{mgH}{kT}}-1}\]
\[C_V=C_V^0+Nk-\frac{N(mgH)^2e^{\frac{mgH}{kT}}}{(e^{\frac{mgH}{kT}}-1)^2}\frac{1}{kT^2}\]
7.21\\
\[Z_1=e^{-\beta\epsilon_1}+e^{-\beta\epsilon_2}\]
令$|\epsilon_2-\epsilon_1|=\Delta,\dfrac{\epsilon_1+\epsilon_2}{2}=\epsilon_0$,我们有
\[Z_1=2e^{-\beta\epsilon_0}\cosh\beta\frac{\Delta}{2}\]
\[U=-N\frac{\partial\ln Z_1}{\partial\beta}=N\epsilon_0-N\frac{\Delta}{2}\tanh\beta\frac{\Delta}{2}\]
\[S=Nk\left(\ln Z_1-\beta\frac{\partial\ln Z_1}{\partial\beta}\right)=Nk\left(\ln 2\cosh\beta\frac{\Delta}{2}-\beta\frac{\Delta}{2}\tanh\beta\frac{\Delta}{2}\right)\]
当$T\rightarrow0,\beta\rightarrow+\infty$时,
\[U\rightarrow N\epsilon_0-N\frac{\Delta}{2},S\rightarrow0\]
这是因为低温极限下系统处于能量最低态,且无简并。\\
当$T\rightarrow+\infty,\beta\rightarrow0$时,
\[U\rightarrow N\epsilon_0,S\rightarrow Nk\ln2\]
这是因为高温极限下系统处于任何态的概率都相等,因此能量为两能级平均值,熵为$Nk\ln2$。\\
7.23\\
\begin{align*}
	Z_1^r&=\frac{1}{h^2}\iint\mathrm{d}p_\theta\mathrm{d}p_\varphi\int_{0}^{\pi}\mathrm{d}\theta\int_{0}^{2\pi}\mathrm{d}\varphi\exp\left\{-\beta\left[\frac{1}{2I}\left(p_\theta^2+\frac{p_\varphi^2}{\sin^2\theta}\right)-d_0E\cos\theta\right]\right\}\\
	&=\frac{I}{\beta\hbar^2}\int_{0}^{\pi}\sin\theta e^{\beta d_0E\cos\theta}\mathrm{d}\theta\\
	&=\frac{I}{\beta\hbar^2}\frac{e^{\beta d_0E}-e^{-\beta d_0E}}{\beta d_0E}
\end{align*}
\end{document}













