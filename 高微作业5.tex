\documentclass[utf8]{ctexart}
\usepackage{graphicx}
\usepackage{amsmath}
\usepackage{amssymb}
\title{高微作业5}
\author{郑子诺,物理41}
\date{\today}
\begin{document}
\maketitle
\noindent1.\\
\[f(x)=x^{2n}+a_{2n-1}x^{2n-1}+\cdots+a_1x+a_0=x^{2n}(1+\frac{a_{2n-1}}{x}+\cdots+\frac{a_0}{x^{2n}})\]
\[\because\lim\limits_{x\rightarrow\infty}1+\frac{a_{2n-1}}{x}+\cdots+\frac{a_0}{x^{2n}}=1\]
\[\therefore\exists N,|x|>N,1+\frac{a_{2n-1}}{x}+\cdots+\frac{a_0}{x^{2n}}>0\]
\[\therefore x>N,f(x)>0;-x<-N,f(x)>0\]
又因为$f(0)=a_0<0$,所以根据介值定理,$[0,N+1],[-N-1,0]$各存在一个零点,所以至少有两个零点。\\
2.\\
由于$\lim\limits_{x\rightarrow\infty}f(x)=+\infty$,所以
\[\forall M,\exists N,|x|>N,f(x)>M\]
因此$(-\infty,-N),(N,+\infty)$内的函数都有下界。对$[-N,N]$内的函数使用有界性定理,存在下界。因此$f(x)$的值域存在下界,因而有下确界$\alpha$。\\
若取不到下确界,令
\[g(x)=\frac{1}{f(x)-\alpha}\]
且取$M=\alpha+\epsilon_0$。则对于$0<\epsilon<\epsilon_0,\exists x\in[-N,N],f(x)-\alpha<\epsilon$。
\[\therefore\forall0<\epsilon<\epsilon_0,\exists x\in[-N,N],g(x)>\frac{1}{\epsilon}\]
与有界性矛盾,因此存在最小值$f(x_0)$。\\
3.\\
利用第一题结论,我们有
\[\exists N,|x|>N,1+\frac{a_3}{x}+\frac{a_2}{x^2}+\frac{a_1}{x}+\frac{a_0}{x^4}>0\]
同时$\lim\limits_{x\rightarrow\infty}x^4=+\infty$。所以
\[\lim\limits_{x\rightarrow\infty}P(x)=+\infty\]
利用第二题结论,即可得$P(x)$存在最小值$P(x_0)$。\\
4.\\
首先我们证明伯努利不等式:对于$x>-1$,我们有
\[(1+x)^\alpha\ge1+\alpha x,\alpha>1\]
\[(1+x)^\alpha\le1+\alpha x,0<\alpha<1\]
\[(1+x)^\alpha\ge1+\alpha x,\alpha<0\]
对于正整数情况,用数学归纳法,$n=1$成立,设$n=k$成立,则有
\[(1+x)^{k+1}\ge(1+kx)(1+x)\ge1+(k+1)x\]
因此成立。
对于有理数$r$,先设$r=\frac{p}{q}<1$,我们有
\[(1+x)^\frac{p}{q}=\sqrt[q]{(1+x)^p}\le\frac{p(1+x)+q-p}{q}=1+\frac{p}{q}x\]
若$r=\frac{p}{q}>1$,则有
\[(1+rx)^\frac{1}{r}\le1+x\rightarrow(1+x)^r\ge1+rx\]
若$r<0$,存在充分大的正整数$n$使得$-1<\frac{-rx}{n}<1$,因此
\[(1+x)^\frac{-r}{n}\le1-\frac{r}{n}x\le\frac{1}{1+\frac{r}{n}x}\]
\[(1+x)^r\ge(1+\frac{r}{n}x)^n\ge1+rx\]
由有理数的稠密性以及指数函数的连续性可推知伯努利不等式对一切实数成立。
回到题目,$\alpha=0,1$显然一致连续。若$\alpha>1$,则有
\[(x+\delta)^\alpha-x^\alpha=x^\alpha((1+\frac{\delta}{x})^\alpha-1)>\alpha\delta x^{\alpha-1}\]
随$x$递增趋于无穷,因此不一致连续。若$0<\alpha<1$,则有
\[(x+\delta)^\alpha-x^\alpha=x^\alpha((1+\frac{\delta}{x})^\alpha-1)<\alpha\delta x^{\alpha-1}\]
随$x$递减,因此取$\delta<\frac{\epsilon}{\alpha}$即可,因而一致连续。若$\alpha<0$,则有
\[x^\alpha-(x+\delta)^\alpha=x^\alpha(1-(1+\frac{\delta}{x})^\alpha)<-\alpha\delta x^{\alpha-1}\]
随$x$递减,因此取$\delta<-\frac{\epsilon}{\alpha}$即可,因而一致连续。\\
综上所述,$\alpha\le1$,一致连续,$\alpha>1$,不一致连续。\\
5.\\
\[\ln x<x^\alpha<a^x<[x]!<x^x\]
首先
\[e^x>(1+\frac{x}{N})^N,N>\alpha\]
\[\lim\limits_{x\rightarrow+\infty}\frac{a^x}{x^\alpha}>\lim\limits_{x\rightarrow+\infty}(1+\frac{x\ln a}{N})^{N-\alpha}(\frac{1}{x}+\frac{\ln a}{N})^\alpha=+\infty\]
其次
\[\lim\limits_{x\rightarrow+\infty}\frac{x^\alpha}{\ln x}=\lim\limits_{t\rightarrow+\infty}\frac{e^t}{t^\frac{1}{\alpha}}=+\infty\]
然后
\[\frac{[x+1]!}{a^{[x+1]}}\frac{a^{[x]}}{[x]!}=\frac{[x+1]}{a}\]
当$x$足够大时有$\frac{[x+1]}{a}>k>1$,因此
\[\lim\limits_{x\rightarrow+\infty}\frac{[x]!}{a^{[x]}}=\lim\limits_{x\rightarrow+\infty}\frac{[x]!}{a^{[x+1]}}=+\infty\]
\[\lim\limits_{x\rightarrow+\infty}\frac{[x]!}{a^x}>\lim\limits_{x\rightarrow+\infty}\frac{[x]!}{a^{[x+1]}}=+\infty\]
最后
\[\frac{[x+1]^{[x+1]}}{[x+1]!}\frac{[x]!}{[x]^{[x]}}=(1+\frac{1}{[x]})^{[x]}\]
因为$\lim\limits_{x\rightarrow+\infty}(1+\frac{1}{x})^x=e$
所以$x$足够大时,$(1+\frac{1}{[x]})^{[x]}>k>1$。因此
\[\lim\limits_{x\rightarrow+\infty}\frac{[x]^{[x]}}{[x]!}=+\infty\]
\[\lim\limits_{x\rightarrow+\infty}\frac{x^x}{[x]!}>\lim\limits_{x\rightarrow+\infty}\frac{[x]^{[x]}}{[x]!}=+\infty\]
6.\\
由题意可知
\[a_{n+1}=f(a_n)<\frac{a_n}{1+a_n}\]
\[\frac{1}{a_{n+1}}>1+\frac{1}{a_n}>n+\frac{1}{a_1}>n+1\]
\[\therefore na_n<1,\lim\limits_{n\rightarrow\infty}a_n=0\]
因此对于每一个$t$,存在$N$使得$n>N$时有
\[a_{n+1}=f(a_n)>\frac{a_n}{1+ta_n}\]
\[\frac{1}{a_{n+1}}<t+\frac{1}{a_n}<nt+\frac{1}{a_1}\]
\[\therefore na_n>\frac{n}{nt-t+\frac{1}{a_1}}\]
\[\because\lim\limits_{n\rightarrow\infty}\frac{n}{nt-t+\frac{1}{a_1}}=\frac{1}{t}\]
而且$t$为任意大于$1$的数。因此存在$N_0$使得$na_n>1-\epsilon,\forall\epsilon>0$。于是我们有
\[\frac{n}{nt-t+\frac{1}{a_1}}na_n=1\]
\end{document}


















